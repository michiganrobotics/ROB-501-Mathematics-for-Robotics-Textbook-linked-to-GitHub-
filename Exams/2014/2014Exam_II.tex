%\documentclass[11pt,twoside]{nsf_jwg} %!PN
\documentclass[letterpaper]{article}
\usepackage{amssymb}
\usepackage[cm]{fullpage}
\usepackage{amsmath}
\usepackage{epsfig,float,alltt}
\usepackage{psfrag,xr}
\usepackage[T1]{fontenc}
\usepackage{url}
\usepackage{pdfpages}
%\includepdfset{pagecommand=\thispagestyle{fancy}}

%
%***********************************************************************
%               New Commands
%***********************************************************************
%
%
\newcommand{\rb}[1]{\raisebox{1.5ex}{#1}}
 \newcommand{\trace}{\mathrm{trace}}
\newcommand{\real}{\mathbb R}  % real numbers  {I\!\!R}
\newcommand{\nat}{\mathbb R}   % Natural numbers {I\!\!N}
\newcommand{\cp}{\mathbb C}    % complex numbers  {I\!\!\!\!C}
\newcommand{\pp}{\mathbb P}      %{I\!\!\!\!P}
\newcommand{\ds}{\displaystyle}
\newcommand{\mf}[2]{\frac{\ds #1}{\ds #2}}
\newcommand{\Luenberger}[2]{{Luenberger, Page~#1, }{Prob.~#2}}
\newcommand{\Nagy}[2]{{Nagy, Page~#1, }{Prob.~#2}}
\newcommand{\spanof}[1]{\textrm{span} \{ #1 \}}
 \newcommand{\cov}{\mathrm{cov}}
 \newcommand{\E}{\mathcal{E}}
 \newcommand{\Expectof}[1]{{\cal E} \{ #1 \}}
  \newcommand{\ExpectofGiven}[2]{{\cal E} \{ #1 | #2 \}}
  \newcommand{\Covof}[2]{ \mathrm{cov} \left(#1,#2\right)}
\parindent 0pt

\newcommand{\bline}[1]{\underline{\hspace*{#1}}}
%
%
%***********************************************************************
%
%               End of New Commands
%
%***********************************************************************
%
%

\begin{document}

%\pagestyle{plain}

\markboth{\bf Place name or initials here:\underline{\hspace*{1.5in}}}{\bf Place name or initials here:\underline{\hspace*{1.5in}}}

\begin{flushright}
{\bf Exam Number:}\bline{0.6in}
\end{flushright}

\vspace*{.1in}
\begin{center}
\LARGE \bf
ROB 501 Exam-II \\
\large
Tuesday, November,18 2014, 6:10 PM -- 7:30 PM \\
Room 1010 DOW \\ %%DOW1010
\end{center}

\vspace*{1in}

\noindent {\bf HONOR PLEDGE:} Copy (NOW) and SIGN ({\bf after the exam is completed}): I have neither given nor received aid on this exam, nor have I observed a violation of the
Engineering Honor Code.

\vspace*{1in}
\begin{flushright}
\underline{\hspace*{2.5in}} \\
SIGNATURE \\
(Sign {\bf after} the exam is completed)
\end{flushright}

\vspace*{1in}

\begin{center}
$\overline{\mathrm ~~LAST~~NAME~ ({\tt PRINTED})~~}^, \hspace*{.4in} \overline{\mathrm ~~FIRST~~NAME~~}$ \\

\end{center}

\vspace*{.45in} \noindent {\bf RULES:}
\begin{enumerate}
\item CLOSED TEXTBOOK
\item CLOSED CLASS NOTES
\item CLOSED HOMEWORK
\item CLOSED HANDOUTS
\item 3  SHEETS OF NOTE PAPER (Front and Back), US Letter Size.
\item NO CALCULATORS, CELL PHONES, PDAs, MP3 PLAYERS, etc.
\end{enumerate}
\vspace*{.4in}


\noindent The maximum possible score is 40. For those problems that allow partial credit, show your work clearly on this booklet.

\newpage

\vspace*{1in}

\begin{center}
\Large
\begin{tabular}{|p{1.2in}|p{1.5in}|}
\hline
\multicolumn{2}{|c|}{\textbf{Record Answers Here}}\\
\hline
 & ~~Your Answer\\
\hline
Problem 1 &   (a)~~(b)~~(c)~~(d)~~\\
\hline
Problem 2 &   (a)~~(b)~~(c)~~(d)~~\\
\hline
Problem 3 &   (a)~~(b)~~(c)~~(d)~~\\
\hline
\end{tabular}
\end{center}

\vspace*{1in}

\begin{center}
\begin{tabular}{|p{2in}|p{1in}|p{1in}|}
\hline
\multicolumn{3}{|c|}{\textbf{Scores (Filled in by Instructor)}}\\
\hline
 & Your Score& Max Score \\
\hline
Problems 1-3 &  &   12\\
\hline
Problem 4 &  &   10\\
\hline
Problem 5 &  &   ~6\\
\hline
Problem 6 &  &   12\\
\hline
& & \\
\hline
\textbf{Total} &  &   $\mathbf{40}$\\
\hline
& & \\
\hline
%Optional Bonus Question&  &   $\pm5$\\
%
\end{tabular}
\end{center}

\newpage

%%%\documentclass[11pt,twoside]{nsf_jwg} %!PN
\documentclass[letterpaper]{article}
\usepackage{amssymb}
\usepackage[cm]{fullpage}
\usepackage{amsmath}
\usepackage{epsfig,float,alltt}
\usepackage{psfrag,xr}
\usepackage[T1]{fontenc}
\usepackage{url}
\usepackage{pdfpages}
%\includepdfset{pagecommand=\thispagestyle{fancy}}

%
%***********************************************************************
%               New Commands
%***********************************************************************
%
%
\newcommand{\rb}[1]{\raisebox{1.5ex}{#1}}
 \newcommand{\trace}{\mathrm{trace}}
\newcommand{\real}{\mathbb R}  % real numbers  {I\!\!R}
\newcommand{\nat}{\mathbb R}   % Natural numbers {I\!\!N}
\newcommand{\cp}{\mathbb C}    % complex numbers  {I\!\!\!\!C}
\newcommand{\pp}{\mathbb P}      %{I\!\!\!\!P}
\newcommand{\ds}{\displaystyle}
\newcommand{\mf}[2]{\frac{\ds #1}{\ds #2}}
\newcommand{\Luenberger}[2]{{Luenberger, Page~#1, }{Prob.~#2}}
\newcommand{\Nagy}[2]{{Nagy, Page~#1, }{Prob.~#2}}
\newcommand{\spanof}[1]{\textrm{span} \{ #1 \}}
 \newcommand{\cov}{\mathrm{cov}}
 \newcommand{\E}{\mathcal{E}}
 \newcommand{\Expectof}[1]{{\cal E} \{ #1 \}}
  \newcommand{\ExpectofGiven}[2]{{\cal E} \{ #1 | #2 \}}
  \newcommand{\Covof}[2]{ \mathrm{cov} \left(#1,#2\right)}
\parindent 0pt

\newcommand{\bline}[1]{\underline{\hspace*{#1}}}
%
%
%***********************************************************************
%
%               End of New Commands
%
%***********************************************************************
%
%

\begin{document}

\vspace*{.1in}
\begin{center}
\LARGE \bf
ROB 501 Exam-I \\
\end{center}

\vspace*{1cm}

\subsection*{Time, Place, Rules and Review Session}

\begin{itemize}
\item The exam is Tuesday October 25, 7:10 PM to 9 PM. We will start passing out exams at 7 PM.
\item The exam rooms are assigned by First Letter of Last Name: \textbf{FXB 1012 (A-Ge) and FXB 1109 (Gr-Z)}
\item If you have a conflict with the exam time, please contact me no later than Tuesday October 18, 4 PM. Also, in your email, please include the nature of your conflict.
\item Rules are exactly those on the first page of Exam I 2015. Note that you cannot use your cell phone as a clock.
\item Review Session: Sunday, Oct 23, 5:10 PM to 6:30 PM in room EECS 1500 (led by Prof. Grizzle) and on Monday Oct 24,  6:10 PM to 7:30 PM, in GGBL 2505 (led by GSI Wubing). These will be Q\&A sessions. We are NOT lecturing. We will do our very best to answer questions that you pose. If there are no questions, we go home! Please bring questions.
    \item We will use at least 50\% of Tuesday's lecture (day of exam) also for Q\&A.
\end{itemize}

\subsection*{Material Covered}

\begin{itemize}
\item From Lecture 1 through Lecture 10 on 6 October.
\item HW 01 through HW 05. This is less material than on the 2015 Exam I.
\item Exam material stops with lecture on 06 October. Orthogonal matrices, Positive Definite Matrices, and RLS will be on the final.
\item You may have to invert by hand a $2 \times 2$ matrix.
\end{itemize}
The exam will \textbf{not} cover:
\begin{itemize}
\item Lagrange multipliers
\item Probability
\end{itemize}

\subsection*{Type of Questions}

\begin{itemize}

\item Your exam will look similar to Exam 1 from 2015; see the CANVAS site. There may be a few more multiple choice questions.


\item I do not have any practice problems other than the posted Exams on the CANVAS site. If I had other problems, I would gladly give them to you. The first year's class did not even have the old exams to look at!

    \item There will probably be ONE proof to give on the exam. When doing a proof, you can use as a fact ANYTHING we have established in lecture or HW.

        \item If you give more than one proof or solution to a problem, you must tell me which one to grade. If you do not tell me which one to grade, I will grade the first one, even if it is wrong and something later is correct. What else can I do? The only reason I mention this is because it has come up in the past.

        \item Everyone always wants to come to me and ask if they have shown enough on the workout problems or the proofs. \textbf{I cannot answer that question.} My best advice is to show your work clearly. Show the steps you are following. You do NOT need to re-derive something we have established in class or HW. You can just state it as a fact and then use it.

\subsection*{Suggested Strategy}

Spend at most 25 minutes initially on the multiple choice questions. Mark the ones you are sure of and move on to the work-out problems. Then come back to the multiple choice questions at the end. Yeah, I know, we all hate multiple choice questions, but it is the only way to have a broad coverage of the material with very few calculations. When you look at them, you will see that our multiple choice questions actually consist of four T/F questions, worth 2 points each. You circle only the responses that are true, and it is never the case that all responses are true nor all are false for a given question. Note also that if you mark all as true (or all as false), then you get no credit whatsoever because it is assumed that you are just guessing. So, not matter what, even if you are guessing, please do mark at least ONE as T and DO NOT MARK ALL as T.

%Last year, the first person done with the exam completed it in 35 minutes, with a score of 78 out of 80 ... and it was a first-year student.

\end{itemize}


\end{document} 
%\begin{center}
%\vspace*{6cm}
%
%{\bf \LARGE Page Intentionally Left Blank}\\
%
%\vspace*{3cm}
%\textbf{Anything written here will not be graded.}
%
%\end{center}

\subsection*{Problems 1 - 3 {\rm (12 points: 3 $\times$ 4)}}

{\bf Instructions.} For each problem, select all of the answers that are correct and enter them in the table on page 2. For each problem, there is at least one answer that is correct and one answer that is incorrect. \textit{You will receive no credit for your response if you either circle all of the answers or none of the answers.}

\vspace{0.5in}


\begin{enumerate}
\setlength{\itemsep}{2.5in}

\item[{\bf 1.}] \textbf{Given:} $C$, an $m \times n$ real matrix with linearly independent columns, $y$ an $m \times 1$ real vector, and  $Q\succ 0$ a real $m \times m$ positive definite matrix. \textbf{Define:}
    $$K=(C^\top Q^{-1} C)^{-1} C^\top Q^{-1} .$$
\begin{enumerate}
\setlength{\itemsep}{.15in}
\renewcommand{\labelenumi}{(\alph{enumi})}
\setlength{\itemsep}{.1in}
\item $\widehat{x}=Ky$ is the Best Linear Unbiased Estimator (BLUE) for $y=Cx + \epsilon$, when $x\in \real^n$ is deterministic, $\epsilon \in \real^m$ is zero mean and has $\Covof{\epsilon}{\epsilon}=Q$.
\item $\widehat{x}=Ky$ is the Minimum Variance Estimator (MVE) for $y=Cx + \epsilon$, when both $x\in \real^n$ and $\epsilon \in \real^m$ are zero mean, and the covariances satisfy $\Covof{x}{x}=0_{n\times n}$, $\Covof{x}{\epsilon}=0_{n\times m}$ and $\Covof{\epsilon}{\epsilon}= Q$.
    \item $\widehat{x}=Ky$ satisfies $||y-C\widehat{x}||=\inf_{m\in M} ||y-m||$, where $M$ is given by the span of the columns of $C$ and the norm of a point $z\in \real^m$ is $||z||=\sqrt{z^\top Q^{-1} z}$
\item The gain $K$ satisfies $KC=I_{n \times n}$.
\end{enumerate}
%\item[{\bf 1.}] Given $A$, an $m \times n$ real matrix with linearly independent columns, $y$ be an $m \times 1$ real vector, and  $Q$ a real $m \times m$ positive definite matrix. Define
%    $$\widehat{x}=(A^\top Q A)^{-1} A Q y.$$
%\begin{enumerate}
%\setlength{\itemsep}{.15in}
%\renewcommand{\labelenumi}{(\alph{enumi})}
%\setlength{\itemsep}{.1in}
%\item $\widehat{x}=\text{arg} \min_{m \in M} \sqrt{(y-m)^\top Q (y-m)}$, where $M$ is given by the span of the columns of $A$.
%\item $\widehat{x}$ is the Best Linear Unbiased Estimator (BLUE) for $y=Ax + \epsilon$, when $x$ is deterministic, $\epsilon$ is zero mean and has covariance $Q$.
%\item $\widehat{x}$ is the Minimum Variance Estimator (MVE) for $y=Ax + \epsilon$, both $x$ and $\epsilon$ are zero mean, uncorrelated, and $\Covof{x,x}=0_{n\times n}$ and $\Covof{\epsilon,\epsilon}= Q$.
%\item The gain $K=(A^\top Q A)^{-1} A Q$ satisfies $KA=I_{n \times n}$.
%\end{enumerate}


\item[{\bf 2.}]  Suppose the columns of the $4 \times 2$ real matrix $A$  are linearly independent. Let $[Q,R]=\texttt{qr}(A,0)$ and $[U,S,V]=\texttt{svd}(A)$ be the outputs for the indicated MATLAB commands, which are identical to how the QR-Factorization and Singular Value Decomposition were presented in lecture.
\begin{enumerate}
\setlength{\itemsep}{.15in}
\renewcommand{\labelenumi}{(\alph{enumi})}
\setlength{\itemsep}{.1in}
\item $R$ is invertible.
\item $R^\top R=S$
\item The columns of $U$ are eigenvectors of $A^\top A$.
%\item The last column of $V$ is the solution to $\widehat{x} = \text{arg}~\min_{x^\top x=1} x^\top A^\top A x.$ [Yes, the question is saying that $\widehat{x}=\texttt{V(:,end)} ]$.
    \item The first column of $V$ is the solution to $\widehat{x} = \text{arg}~\max_{x^\top x=1} x^\top A^\top A x.$ [Yes, the question is saying that $\widehat{x}=\texttt{V(:,1)}$, and note that $\max$ and not  $\min$ is being used].
\end{enumerate}

\item[{\bf 3.}]  We consider three jointly normal random variables $(X,Y,Z)$, with
$$ \mbox{mean}~~\mu = \left[\begin{array}{r} 1\\  2\\  3\end{array} \right] ~~\mbox{and covariance}~~ \Sigma = \left[  \begin{array}{rrr}  2&   0&   1\\  0&   4&   2\\  1&   2&   6\end{array}
\right] $$

\begin{enumerate}
\setlength{\itemsep}{.15in}
\renewcommand{\labelenumi}{(\alph{enumi})}
\setlength{\itemsep}{.1in}
\item The marginal distribution of $Z$ is normal with variance $\sigma_Z^2=6 - [1,~2] \left[  \begin{array}{rr}  2&   0\\  0&   4\end{array}\right]^{-1}\left[\begin{array}{r} 1\\  2\end{array} \right]$
     \item The conditional expectation, $  \ExpectofGiven{X}{Y=y,Z=z} $, is equal to $1-\frac{1}{10}(y-2) + \frac{1}{5}(z-3)$.
\item $X|Z=z$ and $Y|Z=z$  are independent.
\item The random vector $\left[\begin{array}{r} X\\  Y\end{array} \right]|Z=z$ (the vector $[X,~Y]^\top$ conditioned on $Z=z$) has covariance
$$ \left[  \begin{array}{rr}  1&   -\frac{1}{4} \medskip \\  -\frac{1}{4}&  3\end{array}
\right] $$

\end{enumerate}

\end{enumerate}

\newpage

\vspace*{.7in}
\begin{center}
\huge

Partial Credit Section of the Exam

\end{center}



\vspace*{1in}

{\Large  For the next problems, partial credit is awarded and you MUST show your work. Unsupported answers, even if correct, receive zero credit. In other words, right answer, wrong
reason or no reason could lead to no points. If you come to me and ask whether you have written enough, my answer will be,
\begin{center}
\bf ``I do not know'',
\end{center}
 because answering "yes" or "no"  would be unfair to everyone else. If you show the steps you followed in deriving your answer, you'll probably be fine.
  \emph{If something was explicitly derived in lecture, handouts or homework, you do not have to re-derive it. You can state it as a known fact and then use it.} For example, we know that for compatible square matrices, $\det(A B) = \det(A) \det(B)$, so if you need this fact, simply state it and use it.}

\newpage

\noindent {\bf 4. (10 points)} (Be sure to show your work.) Find the range of $a\in \real$ such that the matrix below is positive definite.
$$P= \left[ \begin{array}{rrr}  a&   2a&   3\\  2a&   5a&   2\\  3&   2&  4\end{array}\right].$$ Place your answer in the box.\\ \\

\fbox{\rule[-0.5cm]{0cm}{1cm} The parameter must satisfy:  \hskip2cm ~~}\\

\newpage
\textit{Please show your work for question 4.}
\newpage

\noindent {\bf 5. (6 points)} (Be sure to show your work.)  Place your answer in the box.  Using the standard Euclidean norm\footnote{Yes, $||x||=\sqrt{(x_1)^2 + (x_2)^2 + (x_3)^2}$.}, and methods from lecture, find $x$ of minimum norm that satisfies the equation

$$ \left[ \begin{array}{rrr}  -1&   1 & 2\\  2&  2 & 1 \end{array} \right] x = \left[ \begin{array}{r}  3 \\4\end{array} \right]$$ \\ \\

\fbox{\rule[-0.5cm]{0cm}{0.5cm} \hskip1cm $\widehat{x}= \left[ \begin{array}{r}  \text{} \\ \text{} \\ \text{} \\  \text{} \end{array} \right]$  \hskip1cm ~~}\\

\newpage
\textit{Please show your work for question 5.}
\newpage

\noindent {\bf 6. (12 points)} (Be sure to show your work) This is a deterministic estimation problem. Suppose $x \in \real^2$ is constant, and for $i \ge 1$, $y_i \in \real$ is related to $x$ by
$$y_i = C_i x.$$
We define for $k\ge 1$,
$$ Y_k =\left[ \begin{array}{c} y_1 \\ \vdots \\ y_k \end{array} \right],~   A_k = \left[ \begin{array}{c} C_1 \\ \vdots \\ C_k  \end{array}  \right],$$
and\footnote{Assume that we have checked that the columns of $A_3$ are linearly independent. And yes, from the problem data you see that $S_k=1$.} for $k \ge 3,$ $$ \widehat{x}_k :=\mathrm{arg~min} \sqrt{\sum_{i=1}^k \left(y_i - C_i x \right)^2}=\mathrm{arg~min} \sqrt{(Y_k - A_k x)^\top (Y_k - A_k x)}.$$


\textbf{Determine:}  $\widehat{x}_{4}$, best estimates of $x$ at time $k=4$, given the following data:

\begin{itemize}
\item $\widehat{x}_{3}=[-2, ~ 3]^\top$,
\item $\left( A_{3}^\top A_{3} \right)=\left[ \begin{array}{rr}  1&  0\\  0& 1\end{array} \right].$
\item $C_{4}=[ 2, ~1]$
\item $y_{4}=5$
\end{itemize}

\begin{enumerate}
\setlength{\itemsep}{.15in}
\renewcommand{\labelenumi}{(\alph{enumi})}
\setlength{\itemsep}{.1in}
\item \textbf{6 Points:} State the method you are using and provide relevant equations.
\vspace*{8cm}

\item  \textbf{6 Points:}  Do the calculations and place your answer in the box. Show your work on the next page(s).\\ \\

\fbox{\rule[-0.5cm]{0cm}{1cm} $ \widehat{x}_{4}=$ \hskip2cm ~~}\\
\end{enumerate}



\newpage
\textit{Please show your work for question 6.}
\newpage



\end{document}






