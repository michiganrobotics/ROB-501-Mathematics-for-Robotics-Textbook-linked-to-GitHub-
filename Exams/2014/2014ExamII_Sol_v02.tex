\documentclass[letterpaper]{article}
\usepackage{amssymb}
\usepackage{fullpage}
\usepackage{amsmath}
\usepackage{epsfig,float,alltt}
\usepackage{psfrag,xr}
\usepackage[T1]{fontenc}
\usepackage{url}


\begin{document}

\newcommand{\trace}{\mathrm{trace}}
\newcommand{\real}{\mathbb R}  % real numbers  {I\!\!R}
\newcommand{\nat}{\mathbb R}   % Natural numbers {I\!\!N}
\newcommand{\cp}{\mathbb C}    % complex numbers  {I\!\!\!\!C}
\newcommand{\ds}{\displaystyle}
\newcommand{\mf}[2]{\frac{\ds #1}{\ds #2}}
\newcommand{\book}[2]{{Luenberger, Page~#1, }{Prob.~#2}}
\newcommand{\spanof}[1]{\textrm{span} \{ #1 \}}
  \newcommand{\Covof}[2]{ \mathrm{cov} \left(#1,#2\right)}
\parindent 0pt


\begin{center}
{\large \bf ROB 501 Exam-II Solutions}
\end{center}

\vspace*{1cm}

\begin{enumerate}
%Prob. 1
\item \noindent \textbf{Problem 1:}
The answeres are (a), (c), (d).
\begin{enumerate}
\setlength{\itemsep}{.1in}
\renewcommand{\labelenumi}{(\alph{enumi})}
\item True. From lecture, $\hat{x} = Ky$ where $K = (C^\top Q^{-1}C)^{-1}C^\top P^{-1}.$

\item[(d)] True. When BLUE was developed, the "unbiased" condition arose from $KC = I.$

\item[(b)] False. The MVE reduces to BLUE when $P = \Covof{x}{x}\rightarrow \infty I$, not $0\cdot I$. When $P = 0$, the MVE gain is $K = 0$. We already know that $x = 0$ with certainty, and we therefore ignore the measurements.

\item[(c)] True. This is a weighted least squares problem. $\hat{x} = (C^\top Q^{-1}C)^{-1}C^\top Q^{-1} y$, which equals $Ky$.
\end{enumerate}

%Prob. 2
\item \noindent \textbf{Problem 2:}
The answers are (a) and (d).
\begin{enumerate}
\setlength{\itemsep}{.1in}
\renewcommand{\labelenumi}{(\alph{enumi})}
\item True. One answer is that $R$ gives the representation of the columns of $A$ in the basis defined by the columns of $Q$, and thus $R$ has the same rank as $A$.
$$\therefore R = 2\times 2 ~ \text{and has rank}~ 2 \implies \text{invertible}.$$
A second answer is, we know $R$ is upper triangular. Let write $Q=[v^1, v^2]$, and $A=[A_1, A_2].$ Then $A_1 = R_{11}v^1$, and we have $|R_{11}| = ||A_1|| \neq 0.$ Also, $A_2 = R_{12}v^1+R_{22}v^2$, and if $R_{22} = 0$, then $A_2$ is linearly dependent on $A_1$, which it is not. Hence $R_{22} \neq 0$.
$$\therefore \det(R) = R_{11} R_{22} \neq 0.$$

\item False. $R^\top R$ is square, while $S$ is $4 \times 2$, the same size as $A$.

\item False. The columns of $U$ are e-vectors of $A\cdot A^\top$.

\item True. This is very similar to HW9, Prob 5. See HW2, Prob 7(b).
\end{enumerate}

%Prob. 3
\item \noindent \textbf{Problem 3:}
The answer is (b)
\begin{enumerate}
\setlength{\itemsep}{.1in}
\renewcommand{\labelenumi}{(\alph{enumi})}
\item False. It's variance is simply 6.

\item True. In the handout on Gaussian Random Vectors, we let $X_1 = X$ and $X_2 = [Y, Z]^\top$. We then compute
$$\begin{aligned}E\{X|Y=y, Z=z\} &= 1 + \left[\begin{array}{cc} 0& 1\end{array}\right]\left[\begin{array}{cc} 4 & 2 \\ 2 & 6 \end{array}\right]^{-1}\Big(\begin{bmatrix} y \\ z \end{bmatrix} - \begin{bmatrix} 2 \\ 3 \end{bmatrix}\Big)\\
&= 1 + \left[\begin{array}{cc} 0& 1\end{array}\right](\frac{1}{20})\left[\begin{array}{cc} 6 & -2 \\ -2 & 4 \end{array}\right]\Big(\begin{bmatrix} y \\ z \end{bmatrix} - \begin{bmatrix} 2 \\ 3 \end{bmatrix}\Big)\\
&= 1 - \frac{1}{10}(y - z) + \frac{1}{5}(z - 3)
\end{aligned}$$

\item  and (d) are both False. The random vector $\begin{bmatrix} X \\ Y\end{bmatrix} | Z = z$ has covariance
$$\left[\begin{array}{cc} 2 & 0 \\ 0 & 4 \end{array}\right] - \begin{bmatrix} 1 \\ 2 \end{bmatrix} (\frac{1}{6}) \left[\begin{array}{cc} 1 & 2 \end{array} \right] = \left[\begin{array}{cc} \dfrac{5}{6} & -\dfrac{1}{3} \\[2ex] \dfrac{1}{3} & 3\dfrac{1}{3} \end{array} \right]$$

This does NOT match the answer in (d), and because the matrix is not diagonal, the conditional random variables $X|Z = z$ and $Y|Z = z$ are not independent.
$\hfill \square$
\end{enumerate}

%Prob. 4
\item \noindent \textbf{Problem 4:}
Using Schur complements, 
$P = \left[\begin{array}{cc} A & B\\ B^\top & C \end{array}\right] \succ 0$ if, and only if,
$ A \succ 0~~  \text{and} ~~C-B^\top A^{-1} B \succ 0.$ Applying this result , we require
$$(i)~A= a\left[\begin{array}{cc} 1 & 2 \\ 2 & 5 \end{array} \right] >0 ~\text{and}~ (ii)~ C-B^\top A^{-1} B = 4 - \left[\begin{array}{cc} 3 & 2 \end{array} \right] \left[\begin{array}{cc} a & 2a \\ 2a & 5a \end{array} \right]^{-1} \begin{bmatrix} 3 \\ 2\end{bmatrix} > 0$$
From (i) we have $a>0$.
From (ii) we have
$$4-\frac{1}{a} \left[ \begin{array}{cc} 3 & 2\end{array} \right] \left[\begin{array}{cc} 1 & 2\\ 2 & 5 \end{array}\right]^{-1}\begin{bmatrix}3 \\ 2\end{bmatrix} >0 $$
$$\Updownarrow$$
$$4-\frac{1}{a} \left[ \begin{array}{cc} 3 & 2\end{array} \right] \left[\begin{array}{cc} 5 & -2\\ -2 & 1 \end{array}\right]\begin{bmatrix}3 \\ 2\end{bmatrix} >0 $$
$$\Updownarrow$$
$$4-\frac{25}{a}>0$$
$$\Updownarrow$$
$$a>\frac{25}{4}$$
$$\therefore \boxed{P>0 \iff a > \frac{25}{4}~~\text{and} ~~a > 0\iff a > \frac{25}{4} }$$

\textbf{Second way:}
$P = \left[\begin{array}{cc} A & B\\ B^\top & C \end{array}\right] \succ 0$ if and only if
$ A \succ 0~~  \text{and} ~~C-B^\top A^{-1} B \succ 0.$
We set $A=a$, $B=\left[ \begin{array}{cc} 2a & 3\end{array} \right] $, $C = \left[ \begin{array}{cc} 5a & 2\\ 2 & 4\end{array} \right]$. From $A \succ0$ we have $a>0$. We form 
$$C-B^\top A^{-1} B = \left[ \begin{array}{rr} a & -4 \\ -4 &4 - \frac{9}{a}\end{array} \right]$$
We apply Schur complements again, and check that
$a>0$ (which we already know is required) and $4 - \frac{9}{a} - \frac{16}{a} >0$ (which is a new condition). The latter simplifies to $4 - \frac{25}{a} >0$, and we obtain the same answer as above.\\

\textbf{Still another way:}
$P = \left[\begin{array}{cc} A & B\\ B^\top & C \end{array}\right] \succ 0$ if and only if
$ C \succ 0~~  \text{and} ~~A-BC^{-1} B^\top  \succ 0.$
We set $A= a\left[\begin{array}{cc} 1 & 2 \\ 2 & 5 \end{array} \right]$, $B=\left[ \begin{array}{c} 3 \\ 2 \end{array} \right] $, $C = 4$. We have no problem seeing $C>0$. 
We form
$$A-BC^{-1} B^\top  = \left[ \begin{array}{rr} a - \frac{9}{4} & 2 a - \frac{3}{2} \medskip \\ 2 a - \frac{3}{2} & 5 a - 1\end{array} \right]$$
We apply Schur complements again, and check that
$a>\frac{9}{4}$  and $(5a-1) - (2a - \frac{3}{2})^2 \frac{1}{ a - \frac{9}{4}} = \frac{4 a^2 - 25 a}{4 a - 9}>0$ (which is the hardest one we have seen so far). Because $a > \frac{9}{4}$, the  latter simplifies to $4a - 25 >0$, and we obtain the same answer as above.\\

\textbf{Remark:} At no point did we compute a single e-value.

\newpage
%Prob. 5
\item \noindent \textbf{Problem 5:}
$\hat{x} = A^\top (A\cdot A^\top)^{-1} b$
\medskip\\
Doing the calculations,
$$A = \left[\begin{array}{ccc}-1 & 1 & 2 \\ 2 & 2 & 1\end{array} \right], ~ b = \begin{bmatrix} 3 \\ 4\end{bmatrix}$$

$$A\cdot A^\top = \left[\begin{array}{cc} 6 & 2\\ 2 & 9 \end{array}\right] \Rightarrow (A\cdot A^\top)^{-1} = \frac{1}{50}\left[\begin{array}{cc} 9 & -2 \\ -2 & 6\end{array}\right]$$

$$A^\top (A A\top)^{-1} = \frac{1}{50} \left[ \begin{array}{rr}-13&  14\\  5&  10\\ 16&   2\end{array} \right]   $$

$$\hat{x} = \frac{1}{50}\left[ \begin{array}{r} 17\\ 55\\ 56\end{array} \right]$$

\newpage

%Prob. 6
\item \noindent \textbf{Problem 6:}
\underline{Solution 1}
\begin{enumerate}
\setlength{\itemsep}{.1in}
\renewcommand{\labelenumi}{(\alph{enumi})}
\item The problem is set up for RLS. For $k\geq 3$
$$\begin{aligned}
P_{k+1} &= P_{k} - P_{k} C_{k+1}^\top [1+C_{k+1}P_{k}C_{k+1}^\top]^{-1}C_{k+1}P_{k}\\
K_{k+1} &= P_{k+1}C_{k+1}^\top\\
\widehat{x}_{k+1} &= \hat{x}_k +K_{k+1}(y_{k+1}-C_{k+1}\hat{x}_k)\end{aligned}$$

\item We are given that
$$Q_3 = A_3^\top A_3 = \left[\begin{array}{cc} 1 & 0 \\ 0 & 1 \end{array}\right] \Rightarrow P_3 = Q_3^{-1} = \left[\begin{array}{cc} 1 & 0 \\ 0 & 1 \end{array}\right]$$

$$\begin{aligned} P_4 &= P_3 - P_3 C_4^\top [1+C_4 P_3 C_4^\top]^{-1}C_4 P_3 \\
&= I - I \begin{bmatrix} 2 \\ 1 \end{bmatrix}(1+\left[\begin{array}{cc} 2 & 1\end{array}\right] I \begin{bmatrix} 2 \\ 1 \end{bmatrix})^{-1} \left[\begin{array}{cc} 2 & 1\end{array}\right]I \\
&= \left[\begin{array}{cc} 1 & 0 \\ 0 & 1\end{array}\right] - \frac{1}{6}\left[\begin{array}{cc} 4 & 2 \\ 2 & 1 \end{array}\right] = \left[\begin{array}{cc} \dfrac{1}{3} & -\dfrac{1}{3}\\[2ex]-\dfrac{1}{3} & \dfrac{5}{6}\end{array}\right] \\ K_4 &= P_4 C_4^\top\\
&= \left[\begin{array}{cc} \dfrac{1}{3} & -\dfrac{1}{3}\\[2ex]-\dfrac{1}{3} & \dfrac{5}{6}\end{array}\right]\begin{bmatrix} 2 \\ 1 \end{bmatrix}\\
&= \begin{bmatrix} \dfrac{1}{3} \\[2ex] \dfrac{1}{6}\end{bmatrix}\\
\widehat{x}_4 &= \hat{x_3}+K_4(y_4 - C_4 \hat{x_3})\\
&= \begin{bmatrix} -2\\3\end{bmatrix} + \begin{bmatrix} \dfrac{1}{3} \\[2ex] \dfrac{1}{6}\end{bmatrix} (5-\left[\begin{array}{cc} 2 & 1\end{array}\right]\begin{bmatrix}
-2 \\ 3\end{bmatrix})\\
&= \begin{bmatrix} -2 \\ 3\end{bmatrix} + \begin{bmatrix}
\dfrac{1}{3} \\[2ex] \dfrac{1}{6} \end{bmatrix} (6)\\
&= \begin{bmatrix} 0 \\ 4 \end{bmatrix}
\end{aligned}$$
\end{enumerate}
\underline{Solution 2}
\begin{enumerate}
\setlength{\itemsep}{.1in}
\renewcommand{\labelenumi}{(\alph{enumi})}
\item The problem is set up for RLS. For $k \geq 3$
$$\begin{aligned}
Q_{k+1} &= Q_{k} + C_{k+1}^\top C_{k+1}\\
K_{k+1} &= (Q_{k+1})^{-1}C_{k+1}^\top\\
\hat{x}_{k+1} &= \hat{x}_k +K_{k+1}(y_{k+1}-C_{k+1}\hat{x}_k)\end{aligned}$$

\item We are given that
$$\begin{aligned}
Q_3 &= A_3^\top A_3 = \left[\begin{array}{cc} 1 & 0 \\ 0 & 1 \end{array}\right] \\
Q_4 &= Q_3 + C_4^\top C_4 \\
&= \left[\begin{array}{cc} 1 & 0 \\ 0 & 1\end{array}\right] + \begin{bmatrix} 2 \\ 1 \end{bmatrix} \left[\begin{array}{cc}2 & 1\end{array}\right]\\
&= \left[\begin{array}{cc} 5 & 2 \\ 2 & 2\end{array}\right]
\end{aligned}$$

$$\begin{aligned}Q_4^{-1} &= \frac{1}{6}\left[\begin{array}{cc} 2 & -2 \\ -2 & 5 \end{array}\right]\\
K_4 &= Q_4^{-1} C_4^\top \\
&= \frac{1}{6} \left[\begin{array}{cc} 2 & -2 \\ -2 & 5 \end{array}\right] \begin{bmatrix} 2 \\ 1 \end{bmatrix} \\
&= \frac{1}{6} \begin{bmatrix} 2 \\ 1 \end{bmatrix} = \begin{bmatrix} \dfrac{1}{3} \\[2ex] \dfrac{1}{6}\end{bmatrix}\end{aligned}$$
$$\widehat{x}_4 ~\text{same as before}.$$ $\hfill \square$
\end{enumerate}
\end{enumerate}
\end{document}
