%\documentclass[11pt,twoside]{nsf_jwg} %!PN
\documentclass[letterpaper]{article}
\usepackage{amssymb}
\usepackage[cm]{fullpage}
\usepackage{amsmath}
\usepackage{epsfig,float,alltt}
\usepackage{psfrag,xr}
\usepackage[T1]{fontenc}
\usepackage{url}
\usepackage{pdfpages}
%\includepdfset{pagecommand=\thispagestyle{fancy}}

%
%***********************************************************************
%               New Commands
%***********************************************************************
%
%
\newcommand{\rb}[1]{\raisebox{1.5ex}{#1}}
 \newcommand{\trace}{\mathrm{trace}}
\newcommand{\real}{\mathbb R}  % real numbers  {I\!\!R}
\newcommand{\nat}{\mathbb R}   % Natural numbers {I\!\!N}
\newcommand{\cp}{\mathbb C}    % complex numbers  {I\!\!\!\!C}
\newcommand{\pp}{\mathbb P}      %{I\!\!\!\!P}
\newcommand{\ds}{\displaystyle}
\newcommand{\mf}[2]{\frac{\ds #1}{\ds #2}}
\newcommand{\Luenberger}[2]{{Luenberger, Page~#1, }{Prob.~#2}}
\newcommand{\Nagy}[2]{{Nagy, Page~#1, }{Prob.~#2}}
\newcommand{\spanof}[1]{\textrm{span} \{ #1 \}}
 \newcommand{\cov}{\mathrm{cov}}
 \newcommand{\E}{\mathcal{E}}
 \newcommand{\Expectof}[1]{{\cal E} \{ #1 \}}
  \newcommand{\ExpectofGiven}[2]{{\cal E} \{ #1 | #2 \}}
  \newcommand{\Covof}[2]{ \mathrm{cov} \left(#1,#2\right)}
\parindent 0pt

\newcommand{\bline}[1]{\underline{\hspace*{#1}}}
%
%
%***********************************************************************
%
%               End of New Commands
%
%***********************************************************************
%
%

\begin{document}

%\pagestyle{plain}



\vspace*{.1in}
\begin{center}
\LARGE \bf
Information for ROB 501 Exam-II (Final)\\
\large
It will be an In Class Exam, Just Like Exam 1
\end{center}
\vspace*{1in}

\begin{center}
\LARGE \bf
Date is Thursday, December 17, 2015, 4:10 PM-- 6:00 PM \\
\large
The Rooms are assigned as follows: \\
(First letter of last name)\\
 EECS 1303 (A-K) and EECS 1200 (L-Z) \\
\end{center}
\vspace*{2in}

\noindent \textbf{Remark:} When you receive your exam, please PRINT your name on it and COPY the honor pledge. DO NOT OPEN THE EXAM UNTIL TOLD TO DO SO.



\vspace*{1in} \noindent {\bf RULES:}
\begin{enumerate}
\item CLOSED TEXTBOOK
\item CLOSED CLASS NOTES
\item CLOSED HOMEWORK
\item CLOSED HANDOUTS
\item  3  SHEETS OF NOTE PAPER (Front and Back), US Letter Size. You can write anything you want on your ``cheat sheets''
\item NO CALCULATORS, CELL PHONES, HEADPHONES, SMART WATHCES, etc.
\end{enumerate}
\vspace*{.4in}


\newpage
\subsection*{Material Covered}

\begin{itemize}
\item Lecture 1 through the material on Real Analysis on Dec. 4, 2015, with emphasis on material since Exam 1.
\item May have to invert by hand a $2 \times 2$ matrix.
\item There will be NO questions on compact sets, convex sets, QPs and LPs.
\item HW \#1 through HW \#10, except remove Lagrange multipliers and 'SVD for image processing'; emphasis is definitely on material since Exam 1.
\item Probability at the level of BLUE, MVE, and the handout on Gaussian Random Vectors.
\item Least squares problems of all types (inner product spaces, weighted least squares\footnote{Means the inner product uses a positive definite matrix other than the identity.}, over determined, under determined, RLS, BLUE, MVE). RLS with forgetting factor is NOT on the exam.
\item Matrix Inversion Lemma, Symmetric matrices, Orthogonal matrices,  QR factorizations, SVD, matrix 2-norm (see SVD handout).
\item Positive definite matrices, Schur complement.
\item Kalman filter. There is no question on the Extended Kalman Filter (EKF)
\item Modified Gram Schmidt is NOT on the exam. Proof of the SVD is not on the exam.
\item General probability as covered in lecture on Nov. 10 and the first part of Nov 12 is NOT on the exam.
\item Hermetian matrices are NOT on the exam (present in SVD handout).
\end{itemize}

\subsection*{Type of Questions}

\begin{itemize}

\item Almost identical to Exam 1, except there are 4 workout problems worth 15 points each instead of 3 worth 20 points each. Also, \textbf{there are no written proofs on the final exam.}

    \item When looking at the old exams:
    \begin{itemize}
    \item  Exam II in class has good examples.
    \item  Exam II Take Home, ignore the EKF problem.
    \item  Exam III (Take Home Final), you have to realize that the Multiple Choice Problems are HARDER because the 2014 students had LOTS of time to think about them!
\item  Exam III (Take Home Final) Ignore Problem 4 because I have said that compactness and convexity are NOT on your exam.
\item  Exam III (Take Home Final) Forget about Prob. 5...it is MUCH MUCH too hard for an in class exam.
\item  Exam III (Take Home Final) Prob. 6 is interesting and is a very useful thing to know. Of course, without MATLAB, you cannot work the problem, hence it cannot be on an in class exam.
    \item  Exam III (Take Home Final) Problem 7 parts (a) and (c) are worth studying. Ignore part (b). Keep in mind that your calculations would have to be simple enough that you could do them without a calculator!
        \item  Exam III (Take Home Final)  Ignore Problem 8. Once again, it is a cool thing to know, but cannot really be done without MATLAB.

    \end{itemize}


\end{itemize}




\end{document}
