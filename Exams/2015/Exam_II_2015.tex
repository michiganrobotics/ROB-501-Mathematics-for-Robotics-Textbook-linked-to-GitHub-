%\documentclass[11pt,twoside]{nsf_jwg} %!PN
\documentclass[letterpaper]{article}
\usepackage{amssymb}
\usepackage[cm]{fullpage}
\usepackage{amsmath}
\usepackage{epsfig,float,alltt}
\usepackage{psfrag,xr}
\usepackage[T1]{fontenc}
\usepackage{url}
\usepackage{pdfpages}
%\includepdfset{pagecommand=\thispagestyle{fancy}}

%
%***********************************************************************
%               New Commands
%***********************************************************************
%
%
\newcommand{\rb}[1]{\raisebox{1.5ex}{#1}}
 \newcommand{\trace}{\mathrm{trace}}
\newcommand{\real}{\mathbb R}  % real numbers  {I\!\!R}
\newcommand{\nat}{\mathbb R}   % Natural numbers {I\!\!N}
\newcommand{\cp}{\mathbb C}    % complex numbers  {I\!\!\!\!C}
\newcommand{\pp}{\mathbb P}      %{I\!\!\!\!P}
\newcommand{\ds}{\displaystyle}
\newcommand{\mf}[2]{\frac{\ds #1}{\ds #2}}
\newcommand{\Luenberger}[2]{{Luenberger, Page~#1, }{Prob.~#2}}
\newcommand{\Nagy}[2]{{Nagy, Page~#1, }{Prob.~#2}}
\newcommand{\spanof}[1]{\textrm{span} \{ #1 \}}
 \newcommand{\cov}{\mathrm{cov}}
 \newcommand{\E}{\mathcal{E}}
  \newcommand{\Expectof}[1]{{\cal E} \{ #1 \}}
  \newcommand{\ExpectofGiven}[2]{{\cal E} \{ #1 | #2 \}}
\parindent 0pt

\newcommand{\bline}[1]{\underline{\hspace*{#1}}}
%
%
%***********************************************************************
%
%               End of New Commands
%
%***********************************************************************
%
%

\begin{document}

%\pagestyle{plain}

\begin{flushright}
{\bf Exam Number:}\bline{0.6in}
\end{flushright}

\vspace*{.1in}
\begin{center}
\LARGE \bf
ROB 501 Exam-II (Final)\\
\large
Thursday, December 17, 2015, 4:10 PM-- 6:00 PM \\
Rooms (First letter of last name): EECS 1303 (A-K) and EECS 1200 (L-Z) \\
%%EECS 1303 (seats 74)  and 1200 (seats 78)  has been reserved ROB 501 exam.
\end{center}

\vspace*{1in}

\noindent {\bf HONOR PLEDGE:} Copy (NOW) and SIGN ({\bf after the exam is completed}): I have neither given nor received aid on this exam, nor have I observed a violation of the
Engineering Honor Code.

\vspace*{1in}
\begin{flushright}
\underline{\hspace*{2.5in}} \\
SIGNATURE \\
(Sign {\bf after} the exam is completed)
\end{flushright}

\vspace*{1in}

\begin{center}
$\overline{\mathrm ~~LAST~~NAME~ ({\tt PRINTED})~~}^, \hspace*{.4in} \overline{\mathrm ~~FIRST~~NAME~~}$ \\

\end{center}

\vspace*{.45in} \noindent {\bf RULES:}
\begin{enumerate}
\item CLOSED TEXTBOOK
\item CLOSED CLASS NOTES
\item CLOSED HOMEWORK
\item CLOSED HANDOUTS
\item 3 SHEETS OF NOTE PAPER (Front and Back), US Letter Size.
\item NO CALCULATORS, CELL PHONES, HEADPHONES, SMART WATHCES, etc.
\end{enumerate}
\vspace*{.4in}



\newpage


\vspace*{1in}

\begin{center}
\Large
\begin{tabular}{|p{1.2in}|p{1.5in}|}
\hline
\multicolumn{2}{|c|}{\textbf{Record Answers Here}}\\
\hline
 & ~~Your Answer\\
\hline
Problem 1 &   (a)~~(b)~~(c)~~(d)~~\\
\hline
Problem 2 &   (a)~~(b)~~(c)~~(d)~~\\
\hline
Problem 3 &   (a)~~(b)~~(c)~~(d)~~\\
\hline
Problem 4 &   (a)~~(b)~~(c)~~(d)~~\\
\hline
Problem 5 &   (a)~~(b)~~(c)~~(d)~~\\
\hline
\end{tabular}
\end{center}

\vspace*{1in}

\begin{center}
\huge
\begin{tabular}{|p{2in}|p{1in}|p{1in}|}
\hline
\multicolumn{3}{|c|}{\textbf{Scores (Filled in by Instructor)}}\\
\hline
 & Your Score& Max Score \\
\hline
Problems 1-5 &  &   20\\
\hline
Problem 6 &  &   15\\
\hline
Problem 7 &  &   15\\
\hline
Problem 8 &  &   15\\
\hline
Problem 9 &  &   15\\
\hline
& & \\
\hline
\textbf{Total} &  &   $\mathbf{80}$\\
\hline
& & \\
\hline
%Optional Bonus Question&  &   $\pm5$\\
%
\end{tabular}
\end{center}

\newpage

%%%\documentclass[11pt,twoside]{nsf_jwg} %!PN
\documentclass[letterpaper]{article}
\usepackage{amssymb}
\usepackage[cm]{fullpage}
\usepackage{amsmath}
\usepackage{epsfig,float,alltt}
\usepackage{psfrag,xr}
\usepackage[T1]{fontenc}
\usepackage{url}
\usepackage{pdfpages}
%\includepdfset{pagecommand=\thispagestyle{fancy}}

%
%***********************************************************************
%               New Commands
%***********************************************************************
%
%
\newcommand{\rb}[1]{\raisebox{1.5ex}{#1}}
 \newcommand{\trace}{\mathrm{trace}}
\newcommand{\real}{\mathbb R}  % real numbers  {I\!\!R}
\newcommand{\nat}{\mathbb R}   % Natural numbers {I\!\!N}
\newcommand{\cp}{\mathbb C}    % complex numbers  {I\!\!\!\!C}
\newcommand{\pp}{\mathbb P}      %{I\!\!\!\!P}
\newcommand{\ds}{\displaystyle}
\newcommand{\mf}[2]{\frac{\ds #1}{\ds #2}}
\newcommand{\Luenberger}[2]{{Luenberger, Page~#1, }{Prob.~#2}}
\newcommand{\Nagy}[2]{{Nagy, Page~#1, }{Prob.~#2}}
\newcommand{\spanof}[1]{\textrm{span} \{ #1 \}}
 \newcommand{\cov}{\mathrm{cov}}
 \newcommand{\E}{\mathcal{E}}
 \newcommand{\Expectof}[1]{{\cal E} \{ #1 \}}
  \newcommand{\ExpectofGiven}[2]{{\cal E} \{ #1 | #2 \}}
  \newcommand{\Covof}[2]{ \mathrm{cov} \left(#1,#2\right)}
\parindent 0pt

\newcommand{\bline}[1]{\underline{\hspace*{#1}}}
%
%
%***********************************************************************
%
%               End of New Commands
%
%***********************************************************************
%
%

\begin{document}

\vspace*{.1in}
\begin{center}
\LARGE \bf
ROB 501 Exam-I \\
\end{center}

\vspace*{1cm}

\subsection*{Time, Place, Rules and Review Session}

\begin{itemize}
\item The exam is Tuesday October 25, 7:10 PM to 9 PM. We will start passing out exams at 7 PM.
\item The exam rooms are assigned by First Letter of Last Name: \textbf{FXB 1012 (A-Ge) and FXB 1109 (Gr-Z)}
\item If you have a conflict with the exam time, please contact me no later than Tuesday October 18, 4 PM. Also, in your email, please include the nature of your conflict.
\item Rules are exactly those on the first page of Exam I 2015. Note that you cannot use your cell phone as a clock.
\item Review Session: Sunday, Oct 23, 5:10 PM to 6:30 PM in room EECS 1500 (led by Prof. Grizzle) and on Monday Oct 24,  6:10 PM to 7:30 PM, in GGBL 2505 (led by GSI Wubing). These will be Q\&A sessions. We are NOT lecturing. We will do our very best to answer questions that you pose. If there are no questions, we go home! Please bring questions.
    \item We will use at least 50\% of Tuesday's lecture (day of exam) also for Q\&A.
\end{itemize}

\subsection*{Material Covered}

\begin{itemize}
\item From Lecture 1 through Lecture 10 on 6 October.
\item HW 01 through HW 05. This is less material than on the 2015 Exam I.
\item Exam material stops with lecture on 06 October. Orthogonal matrices, Positive Definite Matrices, and RLS will be on the final.
\item You may have to invert by hand a $2 \times 2$ matrix.
\end{itemize}
The exam will \textbf{not} cover:
\begin{itemize}
\item Lagrange multipliers
\item Probability
\end{itemize}

\subsection*{Type of Questions}

\begin{itemize}

\item Your exam will look similar to Exam 1 from 2015; see the CANVAS site. There may be a few more multiple choice questions.


\item I do not have any practice problems other than the posted Exams on the CANVAS site. If I had other problems, I would gladly give them to you. The first year's class did not even have the old exams to look at!

    \item There will probably be ONE proof to give on the exam. When doing a proof, you can use as a fact ANYTHING we have established in lecture or HW.

        \item If you give more than one proof or solution to a problem, you must tell me which one to grade. If you do not tell me which one to grade, I will grade the first one, even if it is wrong and something later is correct. What else can I do? The only reason I mention this is because it has come up in the past.

        \item Everyone always wants to come to me and ask if they have shown enough on the workout problems or the proofs. \textbf{I cannot answer that question.} My best advice is to show your work clearly. Show the steps you are following. You do NOT need to re-derive something we have established in class or HW. You can just state it as a fact and then use it.

\subsection*{Suggested Strategy}

Spend at most 25 minutes initially on the multiple choice questions. Mark the ones you are sure of and move on to the work-out problems. Then come back to the multiple choice questions at the end. Yeah, I know, we all hate multiple choice questions, but it is the only way to have a broad coverage of the material with very few calculations. When you look at them, you will see that our multiple choice questions actually consist of four T/F questions, worth 2 points each. You circle only the responses that are true, and it is never the case that all responses are true nor all are false for a given question. Note also that if you mark all as true (or all as false), then you get no credit whatsoever because it is assumed that you are just guessing. So, not matter what, even if you are guessing, please do mark at least ONE as T and DO NOT MARK ALL as T.

%Last year, the first person done with the exam completed it in 35 minutes, with a score of 78 out of 80 ... and it was a first-year student.

\end{itemize}


\end{document} 
%\begin{center}
%\vspace*{6cm}
%
%{\bf \LARGE Page Intentionally Left Blank}\\
%
%\vspace*{3cm}
%\textbf{Anything written here will not be graded.}
%
%\end{center}

\newpage



\subsection*{Problems 1 - 5 {\rm (20 points: 5 $\times$ 4)}}

{\bf Instructions.} For each problem, select all of the answers that are correct and enter them in the table on page 2. For each problem, there is at least one answer that is correct (i.e., true) and one answer that is incorrect (i.e., false). \textit{You will receive no credit for your response if you either circle all of the answers or none of the answers.}

\vspace{0.5in}


\begin{enumerate}
\setlength{\itemsep}{5cm}


\item[{\bf 1.}] For (a) and (b), let $M$ be an arbitrary, real symmetric $n \times n$ matrix, $n\ge 2$.
\begin{enumerate}
\setlength{\itemsep}{.15in}
\renewcommand{\labelenumi}{(\alph{enumi})}
\setlength{\itemsep}{.1in}
\item For all $x\in \real^n$, $x^\top M x \ge 0$.
\item There exists a basis for $\real^n$ consisting of e-vectors of $M$.
\item The matrix $M= \left[ \begin{array}{rrr}  1&   2&   3\\  2&   5&   0\\  3&   0&  4\end{array}\right]$ is positive definite.
\item For all $\alpha >3 $, the matrix $M= \left[ \begin{array}{rrr}  \alpha&   2&   3\\  2&   2&   2\\  3&   2&  4\end{array}\right]$ is positive definite.
\end{enumerate}

\vspace*{-2cm}
\item[{\bf 2.}] This problem looks at various min-norm and estimation problems associated with the equation
$$y = Cx + e,$$
where $C$ is an $m \times n$ matrix and both $m$ and $n$ are greater than or equal to one. From the size of the matrix $C$, you easily deduce the dimensions of $y$, $x$ and $e$. [If you read each part carefully, the problem is straightforward.]

%We define $\widehat{x}= (C^\top R C)^{-1} C^\top R y$
\begin{enumerate}
\setlength{\itemsep}{.3cm}
\renewcommand{\labelenumi}{(\alph{enumi})}
\item Assume $x$ and $e$ are deterministic, the columns of $C$ are linearly independent, and the norm on $\real^m$ is ${||y||=\sqrt{y^\top y}}$. Then $\widehat{x}= (C^\top C)^{-1} C^\top y$ satisfies $||y-C\widehat{x}||= \underset{x \in \real^n} \inf ~~ ||y-Cx||.$

    \item Assume $x$ is deterministic and $e$ is identically zero. Assume also that the rows of $C$ are linearly independent, and the norm on $\real^n$ is $||x||=\sqrt{x^\top Q x}$, with $Q >0$. Then $\widehat{x}= (C^\top QC)^{-1}Q C^\top y$ satisfies $||\widehat{x}||= \underset{y=Cx} \inf ~~ ||x||.$

  \item Assume $x$ is deterministic,  $e$ is a zero-mean random vector with covariance $\E\{e e^\top\}=Q >0$, and the columns of $C$ are linearly independent. Then $\widehat{x}= \underbrace{(C^\top Q^{-1}C)^{-1}C^\top Q^{-1}}_{K}~y$ satisfies
      $$\widehat{x}=  \underset{ \tilde{x}=Ky,~KC=I_{n \times n} }{\arg~~\min} ~~\E \{(x-\tilde{x})^\top (x - \tilde{x}) \} =  \underset{ \tilde{x}=Ky,~KC=I_{n \times n} }{\arg~~\min}  ~~\E\{ \sum \limits_{i=1}^n \left(x_i-\tilde{x_i}\right)^2\} .$$

       \item Assume both $x$ and $e$ are zero-mean random vectors with covariances $\E\{x x^\top\}=P>0, $ $\E\{e e^\top\}=Q >0$, and $\E\{x e^\top\}=0$.  If the columns of $C$ are linearly independent, then the minimum variance estimator (MVE) converges to the best linear unbiased estimator (BLUE)  as the uncertainty in $x$ tends to zero, in other words: MVE $\to$ BLUE as $P \to 0$.

\end{enumerate}


%  \item Assume $x$ is deterministic,  $e$ is a zero-mean random vector with covariance $\E\{e e^\top\}=Q >0$, and the columns of $C$ are linearly independent. Then $\widehat{x}= \underbrace{(C^\top Q^{-1}C)^{-1}C^\top Q^{-1}}_{K}~y$ satisfies $KC=I_{n \times n}$ and
%      $$\widehat{x}=  \arg~\underset{ \tilde{x}=Ky}{\min} ~~\E \{(x-\tilde{x})^\top (x - \tilde{x}) \} = \arg~\underset{ \tilde{x}=Ky}{\min} ~~\E\{ \sum \limits_{i=1}^n \left(x_i-\tilde{x_i}\right)^2\} .$$



%
%\item[{\bf 2.}]  Suppose  that $A$  is a real  $4 \times 4$  matrix. Let $[Q,R]=\texttt{qr}(A,0)$ and $[U,S,V]=\texttt{svd}(A)$ be the outputs for the indicated MATLAB commands, which are identical to how the $QR$-Factorization and Singular Value Decomposition were presented in lecture. Suppose moreover that $S=\textrm{diag}(4,3,2,1)$.
%\begin{enumerate}
%\setlength{\itemsep}{.15in}
%\renewcommand{\labelenumi}{(\alph{enumi})}
%\setlength{\itemsep}{.1in}
%%\item $R$ is invertible.
%\item $R^\top R = V S^2 V^\top$. \textbf{Modify: Singular values of R and relate to S...} Also, [false] if A is symmetric, then entrees of S are e-values of A
%\item The columns of $Q$ are e-vectors of $A$.
%\item There exists a matrix $B$ such that $\textrm{rank}(A-B)=3$ and $\underset{{x^\top x=1}}{\max}~x^\top B^\top B x=1$.
%\item The last column of $V$ is the solution to $\widehat{x} = \text{arg}~\underset{x^\top x=1}{\min} x^\top A^\top A x.$ [Yes, the question is saying that $\widehat{x}=\texttt{V(:,end)} ]$.
%    %\item The first column of $V$ is the solution to $\widehat{x} = \text{arg}~\max_{x^\top x=1} x^\top A^\top A x.$ [Yes, the question is saying that $\widehat{x}=\texttt{V(:,1)}$, and note that $\max$ and not  $\min$ is being used].
%\end{enumerate}

\newpage

\item[{\bf 3.}]  Let $({\cal X}, \real, ||\cdot||)$ be a normed space and let $S \subset {\cal X}$ be a nonempty subset of ${\cal X}$. Recall that $\mathring{S}$ is the interior of $S$ and $\bar{S}$ is the closure.
\begin{enumerate}
\setlength{\itemsep}{.15in}
\renewcommand{\labelenumi}{(\alph{enumi})}
\setlength{\itemsep}{.1in}
%\item For $S \subset {\cal X}$ a subset, its interior is $\mathring{S} = \{x\in {\cal X}~|~ \forall ~\epsilon>0,~ B_\epsilon(x) \subset S\}. $
\item Let $(x_k)$ be a sequence converging to a point $x^*\in {\cal X}$, that is $x_k\to x^*$. If $x_k \in \mathring{S}$ for all $k \ge 1$, then $x^* \in {S}.$
\item The closure of $S$ is $\bar{S} = \{x\in {\cal X}~|~ d(x,S)=0\}. $
\item Let $x\in {\cal X}$. If $d(x,\sim S)>0$, then $x\in \mathring{S}$.
\item $\mathring{\overline{S}} = \overline{~\mathring{S}~}. $ (In words, the interior of the closure is equal to the closure of the interior).
\end{enumerate}





\item[{\bf 4.}]  Consider the finite-dimensional vector space $(\real^n, \real )$, with $n\ge 2$.
\begin{enumerate}
\setlength{\itemsep}{.15in}
\renewcommand{\labelenumi}{(\alph{enumi})}
\setlength{\itemsep}{.1in}

\item If the sequence $(x_k)$ converges to $x^*$ in the 1-norm $||\cdot||_1$, then $(x_k)$ also converges to $x^*$ in the 2-norm $||\cdot||_2.$


    \item  Let $||\cdot||$ be an arbitrary norm on $(\real^n, \real )$ and $S \subset \real^n$ a closed subset. Then $S$ is complete.

\item  Let $||\cdot||$ be an arbitrary norm on $(\real^n, \real )$. Consider a function $f:\real^n \to \real$ that is \underline{discontinuous} at the origin $x_0=0$ and a sequence $(x_k)$ that converges to the origin, that is, $x_k \to 0$. Then the following is true: $\exists~\epsilon>0$ such that, $\forall~N < \infty$, $\exists~k \ge N$ satisfying $|f(x_k)| \ge \epsilon$. In other words, the sequence $f(x_k)$ \underline{cannot} converge to $f(0)$.

\item  Let $||\cdot||$ be an arbitrary norm on $(\real^n, \real )$, $M$ a $k$-dimensional subspace with $1 \le k < n$, and $x_0 \in \real^n$. Then there exists a \underline{unique} $x^*\in M$ satisfying $$||x_0-x^*|| = \mathrm{d}(x_0,M).$$


\end{enumerate}


\newpage


\item[{\bf 5.}] We consider three jointly normal random variables $(X,Y,Z)$, with
$$ \mbox{mean}~~\mu = \left[\begin{array}{r} 3\\  2\\  1\end{array} \right] ~~\mbox{and covariance}~~ \Sigma = \left[  \begin{array}{rrr}  2&   1&   1\\  1&   4&   0\\  1&   0&   2\end{array}
\right] $$
\begin{enumerate}
\setlength{\itemsep}{.15in}
\renewcommand{\labelenumi}{(\alph{enumi})}
\setlength{\itemsep}{.1in}
\item $X$ and $Y$ are independent.
\item The conditional random variables $X{\Big|Z=z}$ and $Y{\Big|Z=z}$ are uncorrelated.
\item The conditional mean, $\ExpectofGiven{\left[ \begin{array}{r} X \\  Y \end{array} \right]}{Z=2} = \left[ \begin{array}{r} 3\\  2 \end{array} \right]$.
\item The variance\footnote{For a random variable, meaning a scalar as oposed to a random vector, variance and covariance are the same thing.} of the random variable $W=X + 2 Y + Z$ is 26, that is, $\mathrm{var}(W) = \mathrm{cov}(W)=26.$
\end{enumerate}



\end{enumerate}

\newpage

\begin{center}
\vspace*{6cm}

{\bf \LARGE Page Intentionally Left Blank}\\

\vspace*{3cm}
\textbf{Anything written here will not be graded.}

\textbf{(You can use it for scratch paper.)}

\end{center}


\newpage

\vspace*{.7in}
\begin{center}
\huge

Partial Credit Section of the Exam

\end{center}



\vspace*{1in}

{\Large  For the next problems, partial credit is awarded and you MUST show your work. Unsupported answers, even if correct, receive zero credit. In other words, right answer, wrong
reason or no reason could lead to no points. If you come to me and ask whether you have written enough, my answer will be,
\begin{center}
\bf ``I do not know'',
\end{center}
 because answering "yes" or "no"  would be unfair to everyone else. If you show the steps you followed in deriving your answer, you'll probably be fine.
  \emph{If something was explicitly derived in lecture, handouts or homework, you do not have to re-derive it. You can state it as a known fact and then use it.} For example, we proved that real symmetric matrices have real e-values. So if you need this fact, simply state it and use it.}

%  \newpage
%\vspace*{8cm}
%
%\begin{center}
%{\bf \LARGE Page Intentionally Left Blank}
%
%\end{center}



  \newpage


\noindent {\bf 6. (15 points)} (Place your answer in the box and show your work below.) Consider the (real) inner product space ${({\cal X}, \real, <\cdot, \cdot>)}$  where ${\cal X}$ is the set of $2 \times 2$ real matrices and the inner product of two matrices $A$ and $B$ is ${<A,B>:= \mathrm{tr}(A^\top B)}$. \\

Find $A$ of minimum norm that satisfies $<A,Y_1> = 6$ and $<A,Y_2> = 3$, for\\

$$Y_1=\left[ \begin{array}{rr}1&1\\0&1 \end{array} \right]~~\text{and}~~ Y_2= \left[ \begin{array}{rr}1 & 0\\1&-1 \end{array} \right].$$\\

\noindent \textbf{To make the problem quick to work, you are given:} $<Y_1,Y_2>=0$ and $<Y_2,Y_2>=3$. \\

\fbox{\rule[-1cm]{0cm}{2cm} $A=$ \hskip2cm ~~}\\



\newpage
\textit{Please show your work for question 6.}


\newpage

\noindent {\bf 7. (15 points)}  (Place your answers in the box and show your work below.) Compute the QR factorization of
$$A= \left[  \begin{array}{rr}  1&   2\\  -1 & 0\end{array} \right] .$$

\noindent \textbf{Answer:} \fbox{ \rule[-1cm]{0cm}{2cm}  $Q=$\hskip 4cm ~ and~~ $R=$\hskip4cm ~~}\\ \\

\noindent \textbf{Remark:} You need to show your work. Because $A$ is $2 \times 2$, you may be able to ``guess'' the answer, write it down by ``inspection", or perform ``brute force'' calculations, such as multiplying $Q$ and $R$, equating the product to $A$ and solving for the various terms. Such solutions will earn no points. Of course, once you have computed $Q$ and $R$, it is OK to multiply them out and check that your answer equals $A$. \textit{Either briefly explain your method or document your calculations and you will be fine.}



% \vspace*{.3in}
% \noindent \textbf{Show your calculations below}
\newpage
\textit{Please show your work for question 7.}
\newpage



  \noindent {\bf 8. (15 points)} (Place your answers in the boxes and show your work below.) Consider  the time-invariant linear system
  \begin{align*}
  x_{k+1} &= \underbrace{\left[ \begin{array}{rr} 0 & -3\\ 3 & 0\end{array} \right]}_{A} x_k + \underbrace{\left[ \begin{array}{c} 1 \\ 1\end{array} \right]}_{G} w_k \\
  \\
  y_k &=   \underbrace{\left[ \begin{array}{cc} 1 & 1\end{array} \right]}_{C} x_k + v_k
  \end{align*}

where $w_k$ and $v_k$ are (scalar) zero mean white Gaussian noise processes, with constant covariances.\\

%where, for all $k \ge 0$, $ \ell \ge 0 $,  $\Expectof{v_k}=0$, $\Expectof{w_k}=0$. Moreover, $\Expectof{v_k v_\ell}=\delta_{k \ell}$, $\Expectof{w_k w_l}=2 \delta_{k \ell}$ and $\Expectof{w_k v_\ell}=0$.\\


%%\newcommand{\ExpectofGiven}[2]{{\cal E} \{ #1 | #2 \}}

\noindent \textbf{Data:}  $y_3=2$,  $\widehat{x}_{3|2}:=\ExpectofGiven{x_3}{y_0, y_1, y_2}= \left[ \begin{array}{c} 1 \\ 0\end{array} \right]$ and $P_{3|2}:=\ExpectofGiven{(x_{3}-\widehat{x}_{3|2})(x_{3}-\widehat{x}_{3|2})^\top}{y_0, y_1, y_2} =\left[ \begin{array}{rr} 2 & 1\\ 1 & 2\end{array} \right] $ and for all $k\ge 0$, $R=R_k=\cov\{w_k\}=1$ and $Q=Q_k=\cov\{v_k\}=2$.\\

\noindent \textbf{Determine:}

\begin{enumerate}
\setlength{\itemsep}{.15in}
\renewcommand{\labelenumi}{(\alph{enumi})}
\setlength{\itemsep}{.1in}
\item $\widehat{x}_{3|3}$
\item $P_{3|3}$
\item $P_{4|3}$
\end{enumerate}

\noindent \textbf{Remark:} You can maximize partial credit by summarizing the relevant equations before doing computations. To speed up your solution, \textbf{you are given that} $P_{3|2} C^\top =  \left[ \begin{array}{c} 3 \\ 3\end{array} \right]$,~~ $\left(C P_{3|2} C^\top + Q\right)=8$, ~~and ~$GRG^\top = \left[ \begin{array}{rr} 1 & 1\\ 1 & 1\end{array} \right]$\\ \\


\fbox{\rule[-1cm]{0cm}{2cm} $\widehat{x}_{3|3}=$ \hskip2cm ~~}~~~~~~~~~~\fbox{\rule[-1cm]{0cm}{2cm} $P_{3|3}=$ \hskip2cm ~~}~~~~~~~~~~~\fbox{\rule[-1cm]{0cm}{2cm} $P_{4|3}=$ \hskip2cm ~~}\\




\newpage
\textit{Please show your work for question 8}

\newpage

\noindent {\bf 9. (15 points)}(Place your answer in the box and show your work below.)  Starting from  $x_0= \left[ \begin{array}{c} 1\\ 1\end{array} \right]$, perform one iteration of the Newton-Raphson Algorithm to find a ``better'' solution of the nonlinear equation $y=h(x)$,
$$\underbrace{\left[ \begin{array}{c} 4\\ 2\end{array} \right]}_{y} = \underbrace{\left[ \begin{array}{c} \cos(\pi x_1) + x_2\\ x_1 x_2+(x_1)^3 \end{array} \right]}_{h(x_1,x_2)}.$$

\fbox{\rule[-1cm]{0cm}{2cm} Answer: $=\left[ \begin{array}{c} ~~~\\ ~~~~~~~~~~\\ ~~~\\ ~~~\end{array} \right]$ \hskip1cm ~~}\\

\noindent \textbf{Remark:} You can maximize partial credit by summarizing the relevant equations before doing computations. If you happen to have the modified Newton-Raphson algorithm on your cheat sheet, then take $\epsilon=1$, which gives the standard version of the algorithm. \\

\newpage

\newpage
\textit{Please show your work for question 9.}

\newpage
\vspace*{2cm}

\begin{center}
(Scratch Paper)\\
{\bf \LARGE Page Intentionally Left Blank: Do Not Remove}\\
(If you write anything here, be sure to indicate to which problem it applies.)

\end{center}


%\newpage
%\noindent {\bf Remove carefully. This is your scratch paper.}
\end{document}

\textbf{Un-unsed Problem:}
\noindent {\bf 6. (15 points)} (Place your answer in the box and show your work below.) Determine the Best Linear Unbiased Estimate (BLUE) of $x \in \real^2$ when
$$ y = Cx + \epsilon$$
and
$$y=\left[ \begin{array}{c} 4\\ 8 \\ 0\end{array} \right]~~~C=\left[ \begin{array}{rr}1&1\\0&2\\-1&1\end{array} \right]~~~~Q=\E\{\epsilon \epsilon^\top\}=\left[
\begin{array}{rrr}0.70&-0.40&-0.10\\-0.40&0.80&0.20\\-0.10&0.20&0.30 \end{array} \right].$$\\

\noindent \textbf{Helpful remark:} You are given the following:
 $$ Q^{-1} C C^\top = \left[ \begin{array}{rrr}6&8&2\\6&8&2\\-2&4&6\end{array} \right],~~~~  C^\top Q^{-1}=\left[ \begin{array}{rrr}2&2&-4\\4&4&2\end{array} \right],~\text{and} ~~~~ C C^\top Q^{-1} = \left[  \begin{array}{rrr}6&6&-2\\8&8&4\\2&2&6\end{array}\right].  $$
 You do not need to verify that these are true! If they are useful, use them, and if not useful, ignore them.\\


%\noindent \textbf{Remark:} You can maximize partial credit by summarizing the relevant equations before doing computations.\\
