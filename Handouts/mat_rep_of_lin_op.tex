\documentclass[11pt,twoside]{nsf_jwg} %!PN

%**********************************************************************
%
%    From Franck
%
%**********************************************************************
\newtheorem{propo}{Proposition}

\newtheorem{prop}{Propriety}

\newtheorem{theo}{THEOREM}

\newtheorem{defi}{Definition}

\newtheorem{rem}{Remark}

\newtheorem{lem}{Lemma}

\newtheorem{cor}{Corrolary}

\newtheorem{ex}{Example}

\newtheorem{step}{Step}

\newenvironment{prf}{{\bf Proof. }}{ }

\newenvironment{pr-necess}{{\bf Proof of necessity. }}{ }

\newenvironment{pr-suffi}{{\bf Proof of sufficiency. }}{ }

\newcommand{\noir} {\hfill \rule [-.3em]{.5em}{.5em}}

\newcommand{\nat}{I\!\!N}

\newcommand{\reel}{I\!\!R}

\newcommand{\indic}{1\!\!I}

\newcommand{\beqnum}{\begin{equation}\begin{array}{lcl}}

\newcommand{\eeqnum}{\end{array}\end{equation}}

\newcommand{\beqnom}{\begin{eqnarray}}

\newcommand{\eeqnom}{\end{eqnarray}}

\newcommand{\beq}{\begin{eqnarray*}}

\newcommand{\eeq}{\end{eqnarray*}}

\newcommand{\rbbox}{\hfill \rule [-.3em]{.5em}{.5em}}

\newcommand{\srbbox}{\hfill \rule [-.3em]{.4em}{.4em}}

\newcommand{\bt}{\begin{theo}}

\newcommand{\et}{\end{theo}}

\newcommand{\blem}{\begin{lem}}

\newcommand{\elem}{\end{lem}}

\newcommand{\bcor}{\begin{cor}}

\newcommand{\ecor}{\end{cor}}

\newcommand{\brem}{\begin{rem}}

\newcommand{\erem}{\end{rem}}

\newcommand{\bs}{\begin{step}}

\newcommand{\es}{\end{step}}

\newcommand{\bp}{\begin{propo}}

\newcommand{\ep}{\end{propo}}

\newcommand{\bef}{\begin{figure}}

\newcommand{\enf}{\end{figure}}

\newcommand{\gradvec}{\mbox{$\stackrel{\displaystyle\longrightarrow}
{\mbox{grad}}$}}

\newcommand{\Diff}{\mbox{d}}

\newcommand{\Span}{\mbox{Span}}

\newcommand{\sg}{\mbox{Sign}}

\newcommand{\dist}{\mbox{dist}}

%**********************************************************************
%
%    End of From Franck
%
%**********************************************************************
%

%
%***********************************************************************
%
%   New Commaands From Courtney's Thesis
%
%
%***********************************************************************
%
%
\usepackage{graphicx} % for pdf, bitmapped graphics files
\usepackage{amssymb}
%\usepackage{fullpage}
\usepackage{amsmath}
\usepackage{xspace}
%\usepackage{citesort}
%\usepackage{rotating}
\usepackage{lscape}
%\usepackage{here}
\usepackage{epsfig}
%
\newcommand{\spacing}[1]
  {\renewcommand{\baselinestretch}{#1}\small\normalsize}
%
%
\newcommand{\stdfig}[3]{
\begin{figure}[!hbt]
        \centerline{\epsfig{file=#3,width=4.5in}}
        \caption{#1}
        \label{#2}
\end{figure}
}
%
%
\newcommand{\sizedfig}[4]{
\begin{figure}[!hbt]
        \centerline{\epsfig{file=#3,width=#4}}
        \caption{#1}
        \label{#2}
\end{figure}
}
%
%
%***********************************************************************
%			Side by Side Figures from Brandt
%***********************************************************************
%
\newcommand{\bisizedfig}[6]{
\begin{figure}[!hbt]
\noindent
\begin{minipage}[h]{.46\linewidth}
   \centering
   \epsfig{file=#3,width=\linewidth}
   \caption{#1}
   \label{#2}
\end{minipage}\hfill
\begin{minipage}[h]{.46\linewidth}
   \centering
   \epsfig{file=#6,width=\linewidth}
   \caption{#4}
   \label{#5}
\end{minipage}
\end{figure}
}
%
%***********************************************************************
%				New Commands From the Latex Manual
%***********************************************************************
%
%
\newcommand{\rb}[1]{\raisebox{1.5ex}{#1}}
%
%
%***********************************************************************
%
%     			End of New Commands
%
%***********************************************************************
%

\begin{document}

\baselineskip=24pt  % Enforce double space

%\baselineskip=18pt  % Enforce 1.5 space

\pagestyle{plain}

\title{\Large \bf  ROB 501 Handout: Grizzle}

\author{ {\bf Matrix Representation of a Linear Operator}
}

\maketitle

%\vspace*{-1in}
%\section*{\mbox{}}

\Large

\noindent {\bf Definition:} Let $(\cal X, \cal F)$ and $(\cal Y, \cal F)$ be finite dimensional vector spaces. A function $L: \cal X \rightarrow \cal Y$ is a \textbf{ linear operator} if for each $x$ and $z$ in $\cal X$, and $\alpha$ and $\beta$ in $\cal F$,
$$L(\alpha x + \beta z)= \alpha L(x) + \beta L(z) $$

\vspace*{.3in}

\noindent {\bf Definition:} Let $(\cal X, \cal F)$ and $(\cal Y, \cal F)$ be finite dimensional vector spaces, and let $L: \cal X \rightarrow \cal Y$ be a linear operator. A {\bf matrix representation of $L$} {\it with respect to the bases $\{u\}=\{u^1,\cdots,u^m \}$ for $\cal X$ and  $\{v\}=\{v^1,\cdots,v^n \}$ for $\cal Y$} is a matrix $A$ with coefficients in $\cal F$ satisfying for every $x \in \cal X$,
$$ [L(x)]_v = A [x]_u $$

\noindent {\bf Theorem:} Let $(\cal X, \cal F)$ and $(\cal Y, \cal F)$ be finite dimensional vector spaces, and let $L: \cal X \rightarrow \cal Y$ be a linear operator. Then $L$ has a matrix representation $A$, moreover,
$$ A= [A_1|A_2|\cdots|A_m],$$
where $A_i$, the $i$-th column of $A$, is given by
$$
A_{i}  =  \left[L\left(u^{i}\right)\right]_{\left\{ v\right\} }$$

\vspace*{.3in}


 \textbf{Proof:} Let $x\in\mathcal{X}$, and write $x=\alpha_{1}u^{1}+\ldots+\alpha_{m}u^{m}$ so that
 $$\left[x\right]_{\left\{ u\right\} }=\left[\begin{array}{c}
\alpha_{1}\\
\vdots\\
\alpha_{m}
\end{array}\right].$$ Then
\begin{align*}
L\left(x\right) & =L\left(\alpha_{1}u^{1}+\ldots+\alpha_{m}u^{m}\right)\\
 & =\alpha_{1}L\left(u^{1}\right)+\ldots+\alpha_{m}L\left(u^{m}\right)
 \end{align*}

 \begin{align*}
\left[L\left(x\right)\right]_{\left\{ v\right\} } & =\left[\alpha_{1}L\left(u^{1}\right)+\ldots+\alpha_{m}L\left(u^{m}\right)\right]_{\left\{ v\right\} }\\
 & =\alpha_{1}\left[L\left(u^{1}\right)\right]_{\left\{ v\right\} }+\ldots+\alpha_{m}\left[L\left(u^{m}\right)\right]_{\left\{ v\right\} }\\
 & =\left[\begin{array}{ccc}
\left[L\left(u^{1}\right)\right]_{\left\{ v\right\} } & \cdots & \left[L\left(u^{m}\right)\right]_{\left\{ v\right\} }\end{array}\right]\left[\begin{array}{c}
\alpha_{1}\\
\vdots\\
\alpha_{m}
\end{array}\right]\\
 & =A\left[\begin{array}{c}
\alpha_{1}\\
\vdots\\
\alpha_{m}
\end{array}\right]\\
 & =A\left[x\right]_{\left\{ u\right\} } \end{align*}
\begin{flushright}
 {\bf End of Proof.}
\end{flushright}

\vspace*{.3in}

\noindent {\bf Key Point:} The
$i$-th column of $A$ is given by $A_i =  \left[ \begin{array}{c}
a_{1i} \\ \vdots \\ a_{ni} \end{array} \right]= \left[ L(u^i) \right]_{ \{v^1,\cdots,v^n \} }$

%\vspace*{.3in}

\newpage

\noindent {\bf Useful Exercise:} Let $A$ be an  $n \times n$ matrix with coefficients in a given field ${\cal F}$.\\
\begin{enumerate}

\item Let $e_1, \cdots, e_n$ denote the natural basis for $({\cal F}^n,  {\cal F})$. Show that $A e_i = A_i$, where $A_i$ is the $i$-th column of $A$.

\item Show that $L: {\cal F}^n \to {\cal F}^n$ by, for $x \in {\cal F}^n$, $L(x) = A x,$ is a linear operator.

\item Let $e_1, \cdots, e_n$ denote again the natural basis for $({\cal F}^n,  {\cal F})$. Find the matrix representation of $L$ with respect to  $e_1, \cdots, e_n$. \underline{Note:} In this case, the bases $\{u\}$ and $\{ v \}$ are the same and equal to the natural basis.

\end{enumerate}

\vspace*{.3in}

\noindent {\bf Remark:} The above exercise can be extended to a non-square matrix, say $n \times m$. In that case,  $L: {\cal F}^m \to {\cal F}^n$. If you take the natural bases on both ${\cal F}^m$ and ${\cal F}^n$, you will get the same result as part (3) of the exercise [namely, the matrix representation of $L$ is $A$ itself.]

\newpage

{\Large \bf
\begin{center}
A Worked Example of Matrix Representations
\end{center}
}
\Large

Let $\left(\mathcal{X},\mathcal{F}\right)=\left({\mathbb R}^{2},{\mathbb R}\right)$,
and define $L:{\mathbb R}^{2}\rightarrow{\mathbb R}^{2}$ by $L\left(e_{1}\right)=3e_{1}+4e_{2}$,
$L\left(e_{2}\right)=-e_{1}+6e_{2}$, where $e_{1}=\begin{bmatrix}1\\
0
\end{bmatrix}$ and $e_{2}=\begin{bmatrix}0\\
1
\end{bmatrix}$ are the canonical basis elements.
\begin{enumerate}
\item What is the matrix representation of $L$ with respect to $\left\{ e_{1},e_{2}\right\} $?
\item What is the matrix representation of $L$ with respect to $\left\{ v^{1},v^{2}\right\} $
where $v^{1}=e_{1}+e_{2}$, $v^{2}=3e_{1}-4e_{2}$?
\end{enumerate}
\textbf{Solution:}
\begin{enumerate}
\item Let $A=$matrix representation of $L$. Then the $i^{th}$ column
of $A=\left[L\left(e_{i}\right)\right]_{\left\{ e_{1},e_{2}\right\} }$.\\
$\therefore\left[L\left(e_{1}\right)\right]_{\left\{ e_{1},e_{2}\right\} }=\begin{bmatrix}3\\
4
\end{bmatrix},\,\left[L\left(e_{2}\right)\right]_{\left\{ e_{1},e_{2}\right\} }=\begin{bmatrix}-1\\
6
\end{bmatrix}$\\
$\implies A=\begin{bmatrix}3 & -1\\
4 & 6
\end{bmatrix}$
\item Let $P$ be the change of coordinates from $\left\{ e_{1},e_{2}\right\} $
to $\left\{ v^{1},v^{2}\right\} $, and $\bar{P}$ be the change of
coordinates from $\left\{ v^{1},v^{2}\right\} $ to $\left\{ e_{1},e_{2}\right\} $.
Note that the $i^{th}$ column of $\bar{P}$ is just the representation
of $v^{i}$ in $\left\{ e_{1},e_{2}\right\} $. That is,
\[\bar{P}=\begin{bmatrix}1 & 3\\
1 & -4
\end{bmatrix}.\]
Recall that $\bar{P}=P^{-1}$, so
\[P=\left(\bar{P}\right)^{-1}=\frac{-1}{7}\begin{bmatrix}-4 & -3\\
-1 & 1
\end{bmatrix}=\frac{1}{7}\begin{bmatrix}4 & 3\\
1 & -1
\end{bmatrix}.\]
Therefore, if $\bar{A}$ is the representation of $L$ in $\left\{ v^{1},v^{2}\right\} $,
then
\[\bar{A}=PAP^{-1}=\frac{1}{7}\begin{bmatrix}4 & 3\\
1 & -1
\end{bmatrix}\begin{bmatrix}3 & -1\\
4 & 6
\end{bmatrix}\begin{bmatrix}1 & 3\\
1 & -4
\end{bmatrix}=\frac{1}{7}\begin{bmatrix}-38 & 16\\
-8 & 25
\end{bmatrix}.\]\\
\textbf{Note:} $\bar{P}$ was readily available, not $P$,
as you may have guessed!!
\item Just to check, let's do thing the {}``long way'':\\
\begin{align*}
L\left(v^{1}\right) & =L\left(e_{1}+e_{2}\right)\\
 & =L\left(e_{1}\right)+L\left(e_{2}\right)\\
 & =\left(3e_{1}+4e_{2}\right)+\left(-e_{1}+6e_{2}\right)\\
 & =2e_{1}+10e_{2}\\
L\left(v^{2}\right) & =L\left(3e_{1}-4e_{2}\right)\\
 & =3L\left(e_{1}\right)-4L\left(e_{2}\right)\\
 & =3\left(3e_{1}+4e_{2}\right)-4\left(-e_{1}+6e_{2}\right)\\
 & 13e_{1}-12e_{2}
\end{align*}
$\left[L\left(v^{1}\right)\right]_{\left\{ v^{1},v^{2}\right\} }=?$
To find it, write
\begin{align*}
\begin{bmatrix}2\\
10
\end{bmatrix} & =\bar{a}_{11}\underbrace{\begin{bmatrix}1\\
1
\end{bmatrix}}_{v^{1}}+\bar{a}_{21}\underbrace{\begin{bmatrix}3\\
-4
\end{bmatrix}}_{v^{2}}=\begin{bmatrix}1 & 3\\
1 & -4
\end{bmatrix}\begin{bmatrix}\bar{a}_{11}\\
\bar{a}_{12}
\end{bmatrix}\\
\implies & \begin{bmatrix}\bar{a}_{11}\\
\bar{a}_{12}
\end{bmatrix}=\frac{1}{7}\begin{bmatrix}38\\
-18
\end{bmatrix}
\end{align*}
Similarly
\begin{align*}
\begin{bmatrix}13\\
-12
\end{bmatrix} & =\bar{a}_{12}\begin{bmatrix}1\\
1
\end{bmatrix}+\bar{a}_{22}\begin{bmatrix}3\\
-4
\end{bmatrix}\\
\implies & \begin{bmatrix}\bar{a}_{12}\\
\bar{a}_{22}
\end{bmatrix}=\frac{1}{7}\begin{bmatrix}16\\
25
\end{bmatrix}\\
\therefore\bar{A} & =\frac{1}{7}\begin{bmatrix}38 & 16\\
-8 & 25
\end{bmatrix} %\mqed
\end{align*}
\end{enumerate}



\end{document} 