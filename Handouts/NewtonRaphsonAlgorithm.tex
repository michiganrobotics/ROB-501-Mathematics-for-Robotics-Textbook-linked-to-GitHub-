% --------------------------------------------------------------

% --------------------------------------------------------------

\documentclass[letterpaper]{article}
\usepackage{amssymb}
\usepackage{fullpage}
\usepackage{amsmath}
\usepackage{epsfig,float,alltt}
\usepackage{psfrag,xr}
\usepackage[T1]{fontenc}
\usepackage{url}
\usepackage{pdfpages}
\usepackage{tikz}
\usepackage{empheq}
\usepackage{enumitem}
\newcommand*\circled[1]{\kern-2.5em%
  \put(0,4){\color{white}\circle*{18}}\put(0,4){\circle{16}}%
  \put(-3,0){\color{black}\bfseries\large#1}~~}
\newcommand*\widefbox[1]{\fbox{\hspace{2em}#1\hspace{2em}}}
\newcommand{\differ}[3]{\frac{d#1(#3)}{d#2}}
\newcommand{\norm}[2][x]{\Vert #1 \Vert_{#2}}
\newcommand{\maxi}{\max\limits_{i}}
\newcommand{\sigmai}{\sum_{i=1}^n}
\newcommand{\abox}[1][x_i]{\vert #1\vert}
\newcommand{\Rn}{\mathbb{R}^n}
\newcommand{\R}{\mathbb{R}}
\newcommand{\pardiff}[2][]{\frac{\partial#1}{\partial#2}}
\newcommand{\barx}{\bar{x}}
\newcommand{\limin}[2][\infty]{\lim\limits_{#1\rightarrow#2}}
\newcommand{\ddt}[3][]{\frac{d#1}{d#2}#3}
\newcommand{\intt}[3]{\int^{#1}_{#2}#3}

\date{}
\begin{document}
\baselineskip=48pt
\setlength{\parskip}{.3in}
\setlength{\itemsep}{.3in}

\pagestyle{plain}
% --------------------------------------------------------------
%                         Start here
% --------------------------------------------------------------

%\renewcommand{\qedsymbol}{\filledbox}

{\Large \bf
\begin{center}
ROB 501 Handout: Grizzle \\
\mbox{} \\
Newton Raphson Algorithm
\end{center}
}

\Large

\begin{enumerate}
\setlength{\itemsep}{.2cm}
\item[]
Let $h:\Rn\rightarrow \Rn$ be \underline{continuously differentiable}, and satisfy
$$\quad\quad\quad\boxed{\det\left(\pardiff[h]{x}(x)\right)\neq 0 \quad \forall x\in \R^n}$$ \\

\textbf{\underline{Problem:}} For $y\in \Rn$ fixed, find a solution of $y = h(x)$; i.e, find $x^*\in \Rn$ s.t. $y=h(x^*)$. We note that this is equivalent to $h(x^*)-y=0$. In other words, we are looking for a root of the equation $h(x)-y=0$,\\

\textbf{\underline{Approach:}} Find a convergent sequence $x_k\to x^*$ such that $$\boxed{\limin[k]{\infty}h(x_k)-y=h(x^*)-y=0}$$
that is, $x^*=\limin[k]{\infty}~x_k$ is a root of $h(x)-y=0$ \\

\textbf{\underline{Idea:}} Write $x_{k+1}=x_k+\Delta x_k$. We want
$$h(x_{k+1})-y = h(x_k+\Delta x_k)-y\approx 0.$$
What should $\Delta x_k$ look like ?\\

Apply Taylor's Theorem, to get
$$h(x_k)+\pardiff[h]{x}(x_k)\Delta x_k-y\approx 0$$
$$\therefore \pardiff[h]{x}(x_k)\Delta x_k \approx -(h(x_k)-y)$$
$$\Delta x_k \approx -\left(\pardiff[h]{x}(x_k)\right)^{-1}(h(x_k)-y)$$

Recalling that $x_{k+1}=x_k+\Delta x_k$, we arrive at Newton's Algorithm, given by
$$\boxed{x_{k+1} = x_k-\left(\pardiff[h]{x}(x_k)\right)^{-1}(h(x_k)-y)}$$\\

In practice, the change in $x_k$ given by  $\Delta x_k = -\left(\pardiff[h]{x}(x_k)\right)^{-1}(h(x_k)-y)$ is often too large. Hence, one uses the so-called \underline{Damped Newton Algorithm}

$$\boxed{x_{k+1} = x_k-\epsilon\left(\pardiff[h]{x}(x_k)\right)^{-1}(h(x_k)-y)}$$

where $\epsilon>0$ provides step size control!\\

\textbf{\underline{Remark:}} Looking ahead to our discussion of contraction mappings, a solution of $h(x)-y$ is a \underline{fixed point} of
$$\boxed{P(x)=x-\epsilon\left(\pardiff[h]{x}(x)\right)^{-1}(h(x)-y)}$$
Indeed,
\begin{align*}
x^* & = P(x^*) \\
&\Updownarrow \\
x^*&=x^*-\epsilon\left(\pardiff[h]{x}(x^*)\right)^{-1}(h(x^*)-y)\\
&\Updownarrow \\
0&=-\epsilon\left(\pardiff[h]{x}(x^*)\right)^{-1}(h(x^*)-y)\\
&\Updownarrow \\
0&=(h(x^*)-y).
\end{align*}

It can be shown that $P$ is a \underline{local contraction} on an open ball around a solution of $h(x)-y=0.$

\newpage

\noindent \textbf{Example} Find the solution to the coupled NL equations

$$0 = h(x)=
\left(\begin{array}{c} \mathrm{x_1} + 2\, \mathrm{x_2} - \mathrm{x_1}\, \left(\mathrm{x_1} + 4\, \mathrm{x_2}\right) - \mathrm{x_2}\, \left(4\, \mathrm{x_1} + 10\, \mathrm{x_2}\right) + 3\\ 3\, \mathrm{x_1} + 4\, \mathrm{x_2} - \mathrm{x_1}\, \left(\mathrm{x_1} + 4\, \mathrm{x_2}\right) - \mathrm{x_2}\, \left(4\, \mathrm{x_1} + 10\, \mathrm{x_2}\right) + 4\\ {\sin\!\left(\mathrm{x_3}\right)}^7 + \frac{\cos\!\left(\mathrm{x_1}\right)}{2}\\ {\mathrm{x_4}}^3 - 2\, {\mathrm{x_2}}^2\, \sin\!\left(\mathrm{x_1}\right) \end{array}\right)
$$
\vspace*{1cm}

\textbf{Initial Guess:} $x_0 = \begin{bmatrix}  7\\
     8\\
     9\\
    10\\
    \end{bmatrix} $


We do 16 iterations of Newton's Algorithm (a nonlinear root finding algorithm) and we obtain:

$$x^*=\left(\begin{array}{r} -2.25957308738366677539068499960\\ 1.75957308738366677539068499960\\ 189.50954100613333978330549312824\\ -1.68458069860197189523093013800 \end{array}\right)
$$

\noindent \textbf{And the error is:}

$$ h(x^*) = \begin{bmatrix}
 ~ ~3.6734198 \times 10^{-39} \\
   ~ ~2.9387359\times 10^{-39}\\
 ~ ~1.2765134\times 10^{-38}\\
 -2.5915832\times 10^{-32} \end{bmatrix} $$

\end{enumerate}















\end{document} 