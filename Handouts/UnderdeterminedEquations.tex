%**************************************************************
%References for commands and symbols:
%1. https://en.wikibooks.org/wiki/LaTeX/Mathematics
%2. http://latex.wikia.com/wiki/List_of_LaTeX_symbols
%**************************************************************

\documentclass[letterpaper]{article}
\usepackage{amssymb}
\usepackage{fullpage}
\usepackage{amsmath}
\usepackage{epsfig,float,alltt}
\usepackage{psfrag,xr}
\usepackage[T1]{fontenc}
\usepackage{url}
\usepackage{pdfpages}
%\includepdfset{pagecommand=\thispagestyle{fancy}}

%
%***********************************************************************
%               New Commands
%***********************************************************************
%
%
\newcommand{\rb}[1]{\raisebox{1.5ex}{#1}}
 \newcommand{\trace}{\mathrm{trace}}
\newcommand{\real}{\mathbb R}  % real numbers  {I\!\!R}
\newcommand{\nat}{\mathbb N}   % Natural numbers {I\!\!N}
\newcommand{\whole}{\mathbb Z}    % Integers/whole numbers  {I\!\!\!\!Z}
\newcommand{\cp}{\mathbb C}    % complex numbers  {I\!\!\!\!C}
\newcommand{\rat}{\mathbb Q}    % rational numbers  {I\!\!\!\!Q}

\newcommand{\ds}{\displaystyle}
\newcommand{\mf}[2]{\frac{\ds #1}{\ds #2}}
\newcommand{\book}[2]{{Luenberger, Page~#1, }{Prob.~#2}}
\newcommand{\spanof}[1]{\textrm{span} \{ #1 \}}
 \newcommand{\cov}{\mathrm{cov}}
 \newcommand{\E}{\mathcal{E}}
\parindent 0pt
%
%
%***********************************************************************
%
%               End of New Commands
%
%***********************************************************************
%

\begin{document}


\baselineskip=48pt  % Enforce double space

%\baselineskip=18pt  % Enforce 1.5 space

\setlength{\parskip}{.3in}
\setlength{\itemsep}{.3in}

\pagestyle{plain}



\Large

\begin{center}\textbf{Underdetermined}\end{center}

We suppose the inner product on $\real^n$ is defined by $<x,y> = x^\top Sy$, where $S>0$ is an $n\times n$ positive definite matrix. We denote the corresponding norm by $\|x\|_S := (x^\top S x)^{1/2}$. \\

        Let $Ax = b$, where $x \in \mathbb{R}^n$, $b\in \mathbb{R}^m$, $A=m\times n$, $n>m$, and $\operatorname{rank}(A)=m$. In other words, we are assuming the \underline{rows} of $A$ are linearly independent instead of the columns of $A$ are linearly independent.
        \newline\newline
        \textbf{Def.}~ If $\forall  b_0 \in \mathbb{R}^m, \exists  x_0 \in \mathbb{R}^n,$ such that $b_0 = Ax_0$, $b=Ax$ is \underline{consistent}.
        \newline\newline
        \textbf{Fact:}~ If $\operatorname{\operatorname{rank}}(A)=$ the number of rows, then the equation $b = Ax$ is consistent.
        \newline\newline
        \textbf{Fact:}~ Suppose $x_0$ is such that $b_0 = Ax_0$, and $V=\{x \in \mathbb{R}^n | b = Ax\}$ is the set of solutions. Then, $V = x_0 + \mathcal{N}(A)$, where $\mathcal{N}(A)=\{x \in \mathbb{R}^n | Ax=0\}$ is the null space of $A$. Therefore, $V$ is the translate of a subspace. We can also say that $V$ is an "affine" space.
        \newline\newline
        \textbf{Theorem:}~ If the rows of $A$ are linearly independent, then
        \begin{equation*}
            \hat{x} := \underset{ x \in V} {\operatorname{argmin}} \|x\|_S = \underset{ Ax=b} {\operatorname{argmin}} \|x\|_S = \underset{ Ax = b} {\operatorname{argmin}} (x^\top Sx)^{\frac{1}{2}}
        \end{equation*}
        exists, is unique,  and is given by
        \begin{equation*}
            \hat{x} = S^{-1} A^\top  \left(AS^{-1} A^\top \right)^{-1} b.
        \end{equation*}
        \textbf{Proof is developed in HW.}



\end{document}
