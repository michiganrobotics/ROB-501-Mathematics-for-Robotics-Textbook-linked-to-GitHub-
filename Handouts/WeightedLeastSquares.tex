\documentclass[letterpaper]{article}
\usepackage{amssymb}
\usepackage{fullpage}
\usepackage{amsmath}
\usepackage{epsfig,float,alltt}
\usepackage{psfrag,xr}
\usepackage[T1]{fontenc}
\usepackage{url}
\usepackage{pdfpages}
%\includepdfset{pagecommand=\thispagestyle{fancy}}

%
%***********************************************************************
%               New Commands
%***********************************************************************
%
%
\newcommand{\rb}[1]{\raisebox{1.5ex}{#1}}
 \newcommand{\trace}{\mathrm{trace}}
\newcommand{\real}{\mathbb R}  % real numbers  {I\!\!R}
\newcommand{\nat}{\mathbb R}   % Natural numbers {I\!\!N}
\newcommand{\cp}{\mathbb C}    % complex numbers  {I\!\!\!\!C}
\newcommand{\ds}{\displaystyle}
\newcommand{\mf}[2]{\frac{\ds #1}{\ds #2}}
\newcommand{\book}[2]{{Luenberger, Page~#1, }{Prob.~#2}}
\newcommand{\spanof}[1]{\textrm{span} \{ #1 \}}
 \newcommand{\cov}{\mathrm{cov}}
 \newcommand{\E}{\mathcal{E}}
 \newcommand{\Expectof}[1]{{\cal E} \{ #1 \}}
  \newcommand{\ExpectofGiven}[2]{{\cal E} \{ #1 | #2 \}}
  \newcommand{\Covof}[2]{ \mathrm{cov} \left(#1,#2\right)}
\parindent 0pt
%
%
%***********************************************************************
%
%               End of New Commands
%
%***********************************************************************
%

\begin{document}

\baselineskip=48pt  % Enforce double space

%\baselineskip=18pt  % Enforce 1.5 space

\setlength{\parskip}{.3in}
\setlength{\itemsep}{.3in}

\pagestyle{plain}

{\Large \bf
\begin{center}
Rob 501 Handout: Grizzle \\
Weighted Least Squares
\end{center}
}


\Large

    Let $M$ be an $n \times n$ positive definite matrix $(M \succ0 )$
   % and let the inner product on $\real^n$ be
%    \begin{equation*}
%        \langle x, y \rangle := x^{\top}My.
%    \end{equation*}
    We revisit the over determined system of equations,
     $$A \alpha = b,$$
      where $A = n \times m, n \ge m, {\mathrm rank}(A)=m, \alpha \in \real^m, \textnormal{and } b \in \real^n $. 
      
      We seek $\hat{\alpha}$ such that
    \begin{equation*}
        \| A\hat{\alpha}-b \| = \min_{\alpha \in \real^m} \| A\alpha-b \|
    \end{equation*}
    where $ \| x \|:=\left(x^\top Mx\right)^{1/2}$ and $M>0$.
    \newline\newline
\textbf{Solution:}~ Define an appropriate inner product space $ \mathcal{X}= \real^n, \mathcal{F}=\real, \langle x, y \rangle := x^\top My $ and decompose $A$ into its columns
    $$A= \left[ \begin{array}{c|c|c|c} A_1 & A_2 & \cdots & A_m\\ \end{array} \right] $$
    We seek 
    $$\hat{x} :=\underset{x \in {\rm span} \{A_1, \ldots, A_m\}}{\rm argmin} ||x-b||^2$$

\noindent \textbf{Normal Equations:}
    \begin{align*}
        \hat{x}&=\hat{\alpha}_1 A_1 + \hat{\alpha}_2 A_2 + \cdots + \hat{\alpha}_m A_m\\
        \\
        G^\top \hat{\alpha}&=\beta \textnormal{, with } G = G^\top\\
        \\
        [G^\top]_{ij}&=[G]_{ij}=\langle A_i, A_j \rangle=A_i^\top MA_j=[A^\top MA]_{ij}\\
        \\
        \beta_i&=\langle b, A_i \rangle=b^\top MA_i=A_i^\top Mb=[A^\top Mb]_i.\\
    \end{align*}
    $$ \boxed{\therefore ~~~~~A^\top MA \hat{\alpha}=A^\top Mb}.$$
    \newline\newline
  Because  $ {\rm rank}(A)=m $, its columns are linearly independent and thus the Gram matrix is invertible. Hence, we conclude that
    \begin{equation*}
        \boxed{~~\hat{\alpha}=(A^\top MA)^{-1}A^\top Mb~~}.
    \end{equation*}
    
    \end{document}

\begin{center}
    \textbf{Recursive Least Squares}
\end{center}
\textbf{Model:}
    \begin{align*}
        y_i&=C_ix+e_i,\ i=1, 2, 3, \dotsb\\
        \\
        C_i& \in \real^{m \times n}\\
        i&=\textnormal{time index}\\
        x&=\textnormal{an unknown \underline{constant} vector} \in \real^n\\
        y_i&=\textnormal{measurements} \in \real^m\\
        e_i&=\textnormal{model "mismatch"} \in \real^m
    \end{align*}

    \textbf{Objective 1:}~ Compute a least squared error estimate of $x$ at time $k$, using all available data at time $k$, ($y_1,\dotsb,y_k$)!
    \newline\newline
    \textbf{Objective 2:}~ Discover a computationally attractive form for the answer.
    \newline\newline
    \textbf{Solution:}
    \begin{align*}
        \hat{x}_k:&=\underset{x \in \real^n}{\operatorname{argmin}} \left(\sum\limits_{i=1}^k (y_i-C_ix)^\top S_i(y_i-C_ix)\right)\\
        &=\underset{x \in \real^n}{\operatorname{argmin}} \left(\sum\limits_{i=1}^k e_i^\top S_ie_i\right)
    \end{align*}
    where $S_i=m \times m$ positive definite matrix. ($ S_i>0 $ for all time index $i$)
    \newline\newline
    \textbf{Batch Solution:}
    \newline\newline
    $Y_k=\left[ \begin{array}{c} y_1\\ y_2\\ \vdots \\ y_k \end{array} \right],
    A_k=\left[ \begin{array}{c} C_1\\ C_2\\ \vdots \\ C_k \end{array} \right],
    E_k=\left[ \begin{array}{c} e_1\\ e_2\\ \vdots \\ e_k \end{array} \right]$
    \newline
    $R_k=\left[
    \begin{array}{cc}
        \begin{array}{cc}
            S_1 & \\
            & S_2
        \end{array} &  \mathbf{0} \\
        \mathbf{0} & \begin{array}{cc}
        \ddots & \\
            & S_k
        \end{array}
    \end{array}\right]=diag(S_1,S_2,\dotsb,S_k)>0$
    \newline\newline
    $ Y_k=A_kx+E_k\textnormal{, } [\textnormal{model for } 1 \leq i \leq k ]$
    \newline
    $\|Y_k-A_kx\|^2=\|E_k\|^2:=E_k^\top R_kE_k$
    \newline\newline
    Since $\hat{x}_k$ is the value minimizing the error $\|E_k\|$, which is the unexplained part of the model,
    \begin{equation*}
        \hat{x}_k=\underset{x \in \real^n}{\operatorname{argmin}}\| E_k \|=\underset{x \in \real^n}{\operatorname{argmin}}\| Y_k-A_kx \|,
    \end{equation*}
    which satisfies the Normal Equations $ (A_k^\top R_kA_k)\hat{x}_k=A_k^\top R_kY_k $.
    \newline\newline
    $ \therefore \underline{\hat{x}_k=(A_k^\top R_kA_k)^{-1}A_k^\top R_kY_k} $, which is called a \underline{Batch Solution}.
    \newline\newline
    \textbf{Drawback:}~ $A_k=km \times n$ matrix, and grows at each step!
    \newline\newline
    \textbf{Solution:}~ Find a recursive means to compute $\hat{x}_{k+1}$ in terms of $\hat{x}_k$ and the new measurement $y_{k+1}$!
    \newline\newline
    Normal equations at time $k$, $ (A_k^\top R_kA_k)\hat{x}_k=A_k^\top R_kY_k $, is equivalent to
    $$\left(\sum\limits_{i=1}^k C_i^\top S_iC_i\right)\hat{x}_k=\sum\limits_{i=1}^k C_i^\top S_iy_i.$$
    \newline\newline
    We define
    \begin{equation*}
        M_k=\sum\limits_{i=1}^k C_i^\top S_iC_i
    \end{equation*}
    so that
    \begin{equation*}
        M_{k+1} = M_k+C_{k+1}^\top S_{k+1}C_{k+1}.
    \end{equation*}
    At time $k+1$,
    \begin{equation*}
        \underbrace{(\sum\limits_{i=1}^{k+1} C_i^\top S_iC_i)}_{M_{k+1}}\hat{x}_{k+1}=\sum\limits_{i=1}^{k+1} C_i^\top S_iy_i
    \end{equation*}
    \newline
    or
    \begin{equation*}
        M_{k+1}\hat{x}_{k+1}=\underbrace{\sum\limits_{i=1}^k C_i^\top S_iy_i}_{M_k\hat{x}_k}+C_{k+1}^\top S_{k+1}y_{k+1}.
    \end{equation*}
    \underline{$\therefore M_{k+1}\hat{x}_{k+1}=M_k\hat{x}_k+C_{k+1}^\top S_{k+1}y_{k+1}$}
    \newline\newline
    \textbf{Good start on recursion!} Estimate at time $k+1$ expressed as a linear combination of the estimate at time $k$ and the latest measurement at time $k+1$.
    \newline\newline
    Continuing,
    \begin{equation*}
        \hat{x}_{k+1}=M_{k+1}^{-1}\left[M_k\hat{x}_k+C_{k+1}^\top S_{k+1}y_{k+1}\right].
    \end{equation*}
    Because
    \begin{equation*}
        M_k=M_{k+1}-C_{k+1}^\top S_{k+1}C_{k+1},
    \end{equation*}
    we have
    \begin{equation*}
        \hat{x}_{k+1}=\hat{x}_k+\underbrace{M_{k+1}^{-1}C_{k+1}^\top S_{k+1}}_{\text{Kalman gain}} \underbrace{(y_{k+1}-C_{k+1}\hat{x}_k)}_{\text{Innovations}}.
    \end{equation*}
    \newline
    Innovations  $y_{k+1}-C_{k+1}\hat{x}_k=$ measurement at time $k+1$ minus the "predicted" value of the measurement = "new information".
    \newline\newline
    In a real-time implementation, computing the inverse of $M_{k+1}$ can be time consuming. An attractive alternative can be obtained by applying \underline{the Matrix Inversion} \underline{Lemma}:
    \begin{equation*}
        \left(A+BCD\right)^{-1}=A^{-1}-A^{-1}B\left(DA^{-1}B+C^{-1}\right)^{-1}DA^{-1}
    \end{equation*}
    Now, following the substitution rule as shown below,
    \begin{equation*}
        A\leftrightarrow M_k\ \ \ B\leftrightarrow C_{k+1}^\top\ \ \ C\leftrightarrow S_{k+1}\ \ \ D\leftrightarrow C_{k+1},
    \end{equation*}
    we can obtain that
    \begin{align*}
        M_{k+1}^{-1}&=\left(M_k+C_k^\top S_{k+1}C_{k+1}\right)^{-1}\\
        &=M_k^{-1}-M_k^{-1}C_{k+1}^\top\left[C_{k+1}M_k^{-1}C_{k+1}^\top+S_{k+1}^{-1}\right]^{-1}C_{k+1}M_k^{-1},
    \end{align*}
    which is a recursion for $M_k^{-1}$!
    \newline\newline
    Upon defining
    \begin{equation*}
        P_k=M_k^{-1},
    \end{equation*}
    we have
    \begin{equation*}
        P_{k+1}=P_k-P_kC_{k+1}^\top\left[C_{k+1}P_kC_{k+1}^\top+S_{k+1}^{-1}\right]^{-1}C_{k+1}P_k
    \end{equation*}
    We note that we are now inverting a matrix that is $m\times m$, instead of one that is $n\times n$. Typically, $n>m$, sometimes by a lot!

\end{document} 