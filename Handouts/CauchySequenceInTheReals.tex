\documentclass[letterpaper]{article}
\usepackage{amssymb}
\usepackage{fullpage}
\usepackage{amsmath}
\usepackage{epsfig,float,alltt}
\usepackage{psfrag,xr}
\usepackage[T1]{fontenc}
\usepackage{url}
\usepackage{pdfpages}
%\includepdfset{pagecommand=\thispagestyle{fancy}}

%
%***********************************************************************
%               New Commands
%***********************************************************************
%
%
\newcommand{\rb}[1]{\raisebox{1.5ex}{#1}}
 \newcommand{\trace}{\mathrm{trace}}
\newcommand{\real}{\mathbb R}  % real numbers  {I\!\!R}
\newcommand{\nat}{\mathbb R}   % Natural numbers {I\!\!N}
\newcommand{\cp}{\mathbb C}    % complex numbers  {I\!\!\!\!C}
\newcommand{\ds}{\displaystyle}
\newcommand{\mf}[2]{\frac{\ds #1}{\ds #2}}
\newcommand{\book}[2]{{Luenberger, Page~#1, }{Prob.~#2}}
\newcommand{\spanof}[1]{\textrm{span} \{ #1 \}}
 \newcommand{\cov}{\mathrm{cov}}
 \newcommand{\E}{\mathcal{E}}
 \newcommand{\Expectof}[1]{{\cal E} \{ #1 \}}
  \newcommand{\ExpectofGiven}[2]{{\cal E} \{ #1 | #2 \}}
  \newcommand{\Covof}[2]{ \mathrm{cov} \left(#1,#2\right)}
\parindent 0pt
 \newcommand{\ak}[1]{a_{#1}}
%
%
%***********************************************************************
%
%               End of New Commands
%
%***********************************************************************
%

\begin{document}


\baselineskip=48pt  % Enforce double space

%\baselineskip=18pt  % Enforce 1.5 space

\setlength{\parskip}{.3in}
\setlength{\itemsep}{.3in}

\pagestyle{plain}

{\Large \bf
\begin{center}
Rob 501 Handout: Grizzle \\
A Useful Cauchy Sequence in $(\real, |\cdot|)$
\end{center}
}

%\maketitle

%\vspace*{-1in}
%\section*{\mbox{}}

\Large

\textbf{Proposition} Let $0 \le c <1$ and let $(a_n)$ be a sequence of real numbers satisfying, $\forall~n\ge 1$,
 $$|\ak{n+1}-\ak{n}| \le c | \ak{n}-\ak{n-1}|. $$
 Then $(a_n)$ is Cauchy in $(\real, |\cdot|)$.

\textbf{Proof:}

\textbf{Claim 1:} $\forall ~n\ge 1$, $|\ak{n+1}-\ak{n}| \le c^n |a_1-a_0|$.

\textbf{Proof:} First observe that $$|\ak{3}-\ak{2}|\le c|\ak{2}-\ak{1}| \le c^2|a_1-a_0|.$$ Then complete the proof by induction.

\textbf{Claim 2:} $\forall ~n\ge 1, k \ge 1$, $|\ak{n+k}-\ak{n}| \le \frac{c^n}{1-c} |a_1-a_0|$.

\textbf{Proof:}
\begin{align*}
|\ak{n+k}-\ak{n}| &\le | \ak{n+k}-\ak{n+k-1} + \ak{n+k-1}-\ak{n+k-2} +\cdots \ak{n+1}-\ak{n}| \\
&\le | \ak{n+k}-\ak{n+k-1}| + |\ak{n+k-1}-\ak{n+k-2}| +\cdots |\ak{n+1}-\ak{n}| \\
&\le c^{n+k-1}|a_1-a_0| + c^{n+k-2}|a_1-a_0| + \cdots + c^n |a_1-a_0|  \\
&\le c^n \big( \sum_{i=0}^{k-1} c^i \big)~ |a_1-a_0| \\
&\le c^n \big( \sum_{i=0}^{\infty} c^i \big)~ |a_1-a_0| \\
&\le c^n \big( \frac{1}{1-c} \big) ~ |a_1-a_0|\\
&\le  \frac{c^n }{1-c}~|a_1-a_0|
\end{align*}

\textbf{Claim 3:} $(a_n)$ is Cauchy

\textbf{Proof:} Consider $m$ and $n$. WLOG, suppose $m \ge n$. If $m=n$, then $|a_m-a_n|=0$. Thus assume $m=n+k$ for some $k\ge1$. Then
$$|a_m-a_n|=|a_{n+k}-a_n|\le \frac{c^n}{1-c} |a_1-a_0| \xrightarrow[n\to \infty, m\to \infty~]0,$$
and thus it is Cauchy.


\textbf{Remark:} Because WLOG we could assume $m\ge n$, from $n\to \infty$, we have both $n\to \infty$ and $ m\to \infty$.

\end{document} 