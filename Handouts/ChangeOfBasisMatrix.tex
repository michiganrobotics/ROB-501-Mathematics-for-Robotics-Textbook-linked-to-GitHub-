%**************************************************************
%References for commands and symbols:
%1. https://en.wikibooks.org/wiki/LaTeX/Mathematics
%2. http://latex.wikia.com/wiki/List_of_LaTeX_symbols
%**************************************************************

\documentclass[letterpaper]{article}
\usepackage{amssymb}
\usepackage{fullpage}
\usepackage{amsmath}
\usepackage{epsfig,float,alltt}
\usepackage{psfrag,xr}
\usepackage[T1]{fontenc}
\usepackage{url}
\usepackage{pdfpages}
%\includepdfset{pagecommand=\thispagestyle{fancy}}

%
%***********************************************************************
%               New Commands
%***********************************************************************
%
%
\newcommand{\rb}[1]{\raisebox{1.5ex}{#1}}
 \newcommand{\trace}{\mathrm{trace}}
\newcommand{\real}{\mathbb R}  % real numbers  {I\!\!R}
\newcommand{\nat}{\mathbb N}   % Natural numbers {I\!\!N}
\newcommand{\whole}{\mathbb Z}    % Integers/whole numbers  {I\!\!\!\!Z}
\newcommand{\cp}{\mathbb C}    % complex numbers  {I\!\!\!\!C}
\newcommand{\rat}{\mathbb Q}    % rational numbers  {I\!\!\!\!Q}

\newcommand{\ds}{\displaystyle}
\newcommand{\mf}[2]{\frac{\ds #1}{\ds #2}}
\newcommand{\book}[2]{{Luenberger, Page~#1, }{Prob.~#2}}
\newcommand{\spanof}[1]{\textrm{span} \{ #1 \}}
 \newcommand{\cov}{\mathrm{cov}}
 \newcommand{\E}{\mathcal{E}}
\parindent 0pt
%
%
%***********************************************************************
%
%               End of New Commands
%
%***********************************************************************
%

\begin{document}


\baselineskip=48pt  % Enforce double space

%\baselineskip=18pt  % Enforce 1.5 space

\setlength{\parskip}{.3in}
\setlength{\itemsep}{.3in}

\pagestyle{plain}


\Large

\noindent \textbf{Easy Facts on Representations:}~
    \begin{itemize}
        \item[] 1. Addition of vectors in $(\mathcal{X},\mathcal{F}) \equiv$ Addition of the representations in $(\mathcal{F}^n,\mathcal{F})$.
        \begin{equation*}
            [x+y]_v=[x]_v+[y]_v
        \end{equation*}
        \item[] 2. Scalar multiplication in $(\mathcal{X},\mathcal{F}) \equiv$ Scalar multiplication with the representations in $(\mathcal{F}^n,\mathcal{F})$.
        \begin{equation*}
            [\alpha x]_v=\alpha[x]_v
        \end{equation*}
        \item[] 3. Once a basis is chosen, any n-dimensional vector space $(\mathcal{X},\mathcal{F})$ "looks like" $(\mathcal{F}^n,\mathcal{F})$.
    \end{itemize}
    
    \newpage

\textbf{Change of Basis Matrix:}~ Let $\{u^1, \dotsb, u^n\}$ and $\{\bar{u}^1, \dotsb, \bar{u}^n\}$ be two bases for $(\mathcal{X},\mathcal{F})$. Is there a relation between $[x]_u$ and $[x]_{\bar{u}}$?

\noindent \textbf{Theorem:}~ $\exists$ an invertible matrix $P$, with coefficients in $\mathcal{F}$, such that $\forall x\in(\mathcal{X},\mathcal{F})$, $[x]_{\bar{u}}=P[x]_u$.
    Moreover, $P=\left[P_1|P_2|\dotsb|P_n\right]$ with $P_i=[u^i]_{\bar{u}}\in\mathcal{F}^{n}$ where $P_i$ is the $i^{th}$ column of the matrix $P$ and $[u^i]_{\bar{u}}$ is the representation of $u^i$ with respect to $\bar{u}$.

\noindent \underline{Proof:}~ Let $x=\alpha_1u^1+\dotsb+\alpha_nu^n=\bar{\alpha}_1\bar{u}^1+\dotsb+\bar{\alpha}_n\bar{u}^n$.
    \begin{align*}
        \alpha&=\begin{bmatrix}\alpha_{1}\\
            \alpha_{2}\\
            \vdots\\
            \alpha_{n}
        \end{bmatrix}=[x]_{u}\\
        \bar{\alpha}&=\begin{bmatrix}\bar{\alpha}_{1}\\
            \bar{\alpha}_{2}\\
            \vdots\\
            \bar{\alpha}_{n}
        \end{bmatrix}=[x]_{\bar{u}}
    \end{align*}
    $\bar{\alpha}=[x]_{\bar{u}}=\Big[ \displaystyle\sum_{i=1}^{n}\alpha_{i}u^{i}\Big]_{\bar{u}}=\displaystyle\sum_{i=1}^{n}\alpha_{i}[u^{i}]_{\bar{u}}=\displaystyle\sum_{i=1}^{n}\alpha_{i} P_{i}=P\alpha$.
    \newline
    Therefore, $\bar{\alpha} = P\alpha = P{[}x{]}_{u}$.
    \newline\newline
    Now we need to show that $P$ is invertible:
    \newline
    Define $\bar{P} = [\bar{P}_{1}|\bar{P}_{2}| \dotsb  |\bar{P}_{n}]$ with $\bar{P}_{i}$=$[\bar{u}^{i}]_{u}$.
    \newline
    Do the same calculations and obtain $\alpha=\bar{P}\bar{\alpha}$.
    \newline
    Then, we can obtain that $\alpha=\bar{P}P\alpha$ and $\bar{\alpha}=P\bar{P}\bar{\alpha}$.
    \newline
    Therefore, $P\bar{P} = \bar{P}P = I$.
    \newline
    In conclusion, $\bar{P}$ is the inverse of $P$ ($ \bar{P}=P^{-1}$). $\square$

    \newpage

\noindent \textbf{Example:}~ $\mathcal{X}$=$\{2\times2$ matrices with real coefficients$\}$, $\mathcal{F}=\mathbb{R}$.
    \begin{align*}
        u&=\left\{ \left[\begin{array}{cc}
            1 & 0\\
            0 & 0
            \end{array}\right],\left[\begin{array}{cc}
            0 & 1\\
            0 & 0
            \end{array}\right],\left[\begin{array}{cc}
            0 & 0\\
            1 & 0
            \end{array}\right],\left[\begin{array}{cc}
            0 & 0\\
            0 & 1
            \end{array}\right]\right\}\\
        \bar{u}&=\left\{ \left[\begin{array}{cc}
            1 & 0\\
            0 & 0
            \end{array}\right],\left[\begin{array}{cc}
            0 & 1\\
            1 & 0
            \end{array}\right],\left[\begin{array}{cc}
            0 & 1\\
            -1 & 0
            \end{array}\right],\left[\begin{array}{cc}
            0 & 0\\
            0 & 1
        \end{array}\right]\right\}
    \end{align*}
   \textbf{ We have following relations:}
    \newline
    $$\alpha=P\bar{\alpha}, P_i=[u^i]_{\bar{u}},~~~ \bar{\alpha}=\bar{P}\alpha, \bar{P}_i=[\bar{u}^i]_u$$ 
    $$\bar{P}^{-1}=P, P^{-1}=\bar{P}$$
    \newline
    
    \textbf{Typically, compute the easier of $P$ or $\bar{P}$, and compute the other by inversion.}
    For this example, we choose to compute $\bar{P}$
    \begin{align*}
        \bar{P}_{1}&=[\bar{u}^{1}]_{u}=\left[\begin{array}{c}
            1\\
            0\\
            0\\
            0
        \end{array}\right]\\
        \bar{P}_{2}&=[\bar{u}^{2}]_{u}=\left[\begin{array}{c}
            0\\
            1\\
            1\\
            0
        \end{array}\right]\\
        \bar{P}_{3}&=[\bar{u}^{3}]_{u}=\left[\begin{array}{c}
            0\\
            1\\
            -1\\
            0
        \end{array}\right]\\
        \bar{P}_{4}&=[\bar{u}^{4}]_{u}=\left[\begin{array}{c}
            0\\
            0\\
            0\\
            1
        \end{array}\right]
    \end{align*}
    Therefore, $\bar{P}$=$\left[\begin{array}{cccc}
        1 & 0 & 0 & 0\\
        0 & 1 & 1 & 0\\
        0 & 1 & -1 & 0\\
        0 & 0 & 0 & 1
    \end{array}\right]$ and $P=\bar{P}^{-1} = \left[\begin{array}{rrrr}
        1 & 0 & 0 & 0\\
        0 & .5 & .5 & 0\\
        0& .5 & -.5 & 0\\
        0 & 0 & 0 & 1
    \end{array}\right]$    


%    1.0000         0         0         0
%         0    0.5000    0.5000         0
%         0    0.5000   -0.5000         0
%         0         0         0    1.0000
%
    \noindent \textbf{What if we did it the other direction?}
   
    \begin{align*}
       {P}_{1}&=[{u}^{1}]_{ \bar{u} }=\left[ \begin{array}{c}
            1\\
            0\\
            0\\
            0
        \end{array}\right]   \leftrightarrow \left[ \begin{array}{cc}
            1 & 0\\
            0 & 0
            \end{array} \right] = 1 \cdot  \left[\begin{array}{cc}
            1 & 0\\
            0 & 0
            \end{array} \right]+ 0 \cdot \left[\begin{array}{cc}
            0 & 1\\
            1 & 0
            \end{array} \right] + 0 \cdot \left[\begin{array}{cc}
            0 & 1\\
            -1 & 0
            \end{array} \right] + 0 \cdot \left[\begin{array}{cc}
            0 & 0\\
            0 & 1
        \end{array} \right]
\\
        {P}_{2}&=[{u}^{2}]_{ \bar{u}}=\left[\begin{array}{c}
            0\\
            .5\\
           .5\\
            0
        \end{array}\right] \leftrightarrow \left[ \begin{array}{cc}
            0& 1\\
            0 & 0
            \end{array} \right] = 0 \cdot  \left[\begin{array}{cc}
            1 & 0\\
            0 & 0
            \end{array} \right]+ 0.5  \left[\begin{array}{cc}
            0 & 1\\
            1 & 0
            \end{array} \right] + .5  \left[\begin{array}{cc}
            0 & 1\\
            -1 & 0
            \end{array} \right] + 0 \cdot \left[\begin{array}{cc}
            0 & 0\\
            0 & 1
        \end{array} \right] \\
        {P}_{3}&=[{u}^{3}]_{ \bar{u}}=\left[\begin{array}{c}
            0\\
           .5\\
            -.5\\
            0
        \end{array}\right] \leftrightarrow \left[ \begin{array}{cc}
            0& 0\\
            1 & 0
            \end{array} \right] = 0 \cdot  \left[\begin{array}{cc}
            1 & 0\\
            0 & 0
            \end{array} \right]+ 0.5  \left[\begin{array}{cc}
            0 & 1\\
            1 & 0
            \end{array} \right] - .5 \left[\begin{array}{cc}
            0 & 1\\
            -1 & 0
            \end{array} \right] + 0 \cdot \left[\begin{array}{cc}
            0 & 0\\
            0 & 1
        \end{array} \right]  \\
        {P}_{4}&=[{u}^{4}]_{ \bar{u}}=\left[\begin{array}{c}
            0\\
            0\\
            0\\
            1
        \end{array}\right] \leftrightarrow \left[ \begin{array}{cc}
            0 & 0\\
            0 & 1
            \end{array} \right] = 0 \cdot  \left[\begin{array}{cc}
            1 & 0\\
            0 & 0
            \end{array} \right]+ 0 \cdot \left[\begin{array}{cc}
            0 & 1\\
            1 & 0
            \end{array} \right] + 0 \cdot \left[\begin{array}{cc}
            0 & 1\\
            -1 & 0
            \end{array} \right] + 1 \cdot \left[\begin{array}{cc}
            0 & 0\\
            0 & 1
        \end{array} \right]
    \end{align*}
    
      Therefore, $P= \left[\begin{array}{rrrr}
        1 & 0 & 0 & 0\\
        0 & .5 & .5 & 0\\
        0& .5 & -.5 & 0\\
        0 & 0 & 0 & 1
    \end{array}\right]$    and $\bar{P} ={P}^{-1} =\left[\begin{array}{rrrr}
        1 & 0 & 0 & 0\\
        0 & 1 & 1 & 0\\
        0 & 1 & -1 & 0\\
        0 & 0 & 0 & 1
    \end{array}\right]$

\end{document}
