% $RCSfile: handout_in_EECS_560.tex,v $
% $Revision: 1.1 $
% $Date: 1999/12/22 13:11:28 $

%\documentclass{article}
\documentclass[11pt,twoside]{nsf_jwg}

%%%%%%%%%%%%%%%%%%%%%%%%%%%%%%%%% PREAMBLE %%%%%%%%%%%%%%%%%%%%%%%%%%%%%%%%%%

% Packages used
\usepackage{amssymb}
\usepackage{amsmath}
\usepackage{alltt}
%\usepackage{psboxit}

\newenvironment{definition}{\vspace*{.3in}\noindent{\bf Definition: }}{ }
\newenvironment{example}{\vspace*{.3in}\noindent{\bf Example: }}{ }
\newenvironment{examples}{\vspace*{.3in}\noindent{\bf Examples: }}{ }
\newenvironment{theorem}{\vspace*{.3in}\noindent{\bf Theorem: }}{ }
\newenvironment{lemma}{\vspace*{.3in}\noindent{\bf Lemma: }}{ }
\newenvironment{corollary}{\vspace*{.3in}\noindent{\bf Corollary: }}{ }
\newenvironment{problem}{\vspace*{.3in}\noindent{\bf Problem:}}{ }
\newenvironment{claim}{\vspace*{.3in}\noindent{\bf Claim:}}{ }
\newenvironment{solution}{\vspace*{.1in}\noindent{\bf Solution: }}{ }
\newenvironment{prooff}{\vspace*{.1in}\noindent{\bf Proof: }}{ }
%\def\QEDD{\fbox{\begin{minipage}{.1in}
%\hspace{.008in}
%\vspace{.02in}
%\end{minipage}} }
\def\QEDD{$\blacksquare$}
\def\prooff{\vspace*{.2in}\noindent{\bf Proof: }}
\def\endprooff{\hspace*{\fill}~\QEDD\par\endtrivlist\unskip}


% The set of real numbers and complex numbers
% $\RR$ and $\CC$ will produce boldfaced R and C, respectively
\newcommand{\RR}{{\mathbb{R}}}
\newcommand{\CC}{{\mathbb{C}}}
\newcommand{\Span}{\mbox{Span}}

% Changing the item numbering of the outermost level of `enumerate'
% environment from `1', `2', ...  to `(a)', `(b)', ...
\renewcommand{\theenumi}{\alph{enumi}}
\renewcommand{\labelenumi}{(\theenumi)}

%%%%%%%%%%%%%%%%%%%%%%%%%%%%%% END OF PREAMBLE %%%%%%%%%%%%%%%%%%%%%%%%%%%%%%

\begin{document}

%%%%%%%%%%%%%%%%%%%%%%%%%%%%%%%%% THE TITLE %%%%%%%%%%%%%%%%%%%%%%%%%%%%%%%%%

	\title{\Large \bf ROB 501 Handout: Grizzle}
	\author{{\bf Gram Schmidt Process}}
	\date{}
	\maketitle
	
	\Large


%%%%%%%%%%%%%%%%%%%%%%%%%%%%%%%%%%%%%%%%%%%%%%%%%%%%%%%%%%%%%%%%%%%%%%%%%%%%%
\noindent \textbf{Given:} $({\cal X, F}, <\cdot, \cdot>)$ an inner product space.\\


\noindent \textbf{Proposition:}
Let $\{ y^1, \cdots, y^k\}$ be a linearly independent set and suppose the set
$\{ v^1, \cdots, v^{k-1} \}$ is orthogonal and
 $$\mbox{span}\{ v^1, \cdots, v^{k-1} \} = \mbox{span}\{ y^1, \cdots, y^{k-1} \} .$$
 Define
 \begin{equation}
	v^k = y^k - \sum_{j=1}^{k-1}
	\frac{<y^k,v^j>}{\| v^j \|^2} \cdot v^j		\label{gram_schmidt}
\end{equation}
where $ \| v^j \|^2 = <v^j,v^j>$. Then, $\{ v^1, \cdots, v^{k} \}$ is orthogonal and
 $$\mbox{span}\{ v^1, \cdots, v^{k} \} = \mbox{span}\{ y^1, \cdots, y^{k} \} $$


 \noindent \textbf{Proof:} The orthogonality is by construction. Indeed, one writes
 $$v^k = y^k - \sum_{j=1}^{k-1} a_{kj}v^j	 $$
 and then $<v^k,v^j>=0$ for $1 \le j < k$  if, and only if,
 $$ a_{kj}=	\frac{<y^k,v^j>}{\| v^j \|^2} .$$


 We next show that $y^k \in \mbox{span}\{ v^1, \cdots, v^k \}$ and $v^k \in \mbox{span}\{ y^1, \cdots, y^k \}$.\\

From (\ref{gram_schmidt}),
\[
	y^k = v^k + \sum_{j=1}^{k-1}
	\frac{<y^k,v^j>}{\| v^j \|^2} \cdot v^j \ \Rightarrow \
	y^k \in
		\mbox{span}\{ v^1, \cdots, v^k \}.
\]
Left to show: $v^k \in \mbox{span}\{ y^1, \cdots, y^k \}$. By hypothesis,
\[
	v^j \in \mbox{span}\{ y^1, \cdots, y^{k-1} \}
		\mbox{ for all } 1 \leqslant j \leqslant k-1,
\]
so
\[
	\sum_{j=1}^{k-1}
	\left( \frac{<v^j,y^k>}{\| v^j \|^2} \right) v^j \
		\in \ \mbox{span}\{ y^1, \cdots, y^{k-1} \} \
		\subset \ \mbox{span}\{ y^1, \cdots, y^k \}.
\]
Clearly, $y^k \in \mbox{span}\{ y^1, \cdots, y^k \}$.
\[
	\therefore v^k = y^k - \sum_{j=1}^{k-1}
	\left( \frac{<v^j,y^k>}{\| v^j \|^2} \right) v^j \
		\in \ \mbox{span}\{ y^1, \cdots, y^k \}
\]
because $\mbox{span}\{ y^1, \cdots, y^k \}$ is a subspace. \hfill  $\blacksquare$


\newpage

\noindent
\underline{Inner Products: Example Computation for $(\RR^3,\RR)$}\\
\textbf{Given data:}
\begin{gather*}
	<p,q> = p^T q = \sum_{i=1}^3 p_i q_i \\
	\{ y^1, y^2, y^3 \} = \left\{
		\left[ \begin{array}{c} 1 \\ 1 \\ 0 \end{array} \right],
		\left[ \begin{array}{c} 1 \\ 2 \\ 3 \end{array} \right],
		\left[ \begin{array}{c} 0 \\ 1 \\ 1 \end{array} \right]
		\right\}
\end{gather*}
\textbf{Apply Gram-Schmidt to Produce an Orthogonal Basis:}
	\begin{align*}
		v^1 &= y^1 = \left[ \begin{array}{c} 1 \\ 1 \\ 0 \end{array} \right]\\
		\quad \| v^1 \|^2 &= (v^1)^T v^1 = 2 ;\\
		\\
		v^2	&= y^2 - \frac{<v^1,y^2>}{\| v^1 \|^2} v^1 \\
		&= \left[ \begin{array}{c} 1 \\ 2 \\ 3 \end{array} \right]
		- \underbrace{%
		   \left[ \begin{array}{ccc} 1 & 1 & 0 \end{array} \right]
		   \left[ \begin{array}{c} 1 \\ 2 \\ 3 \end{array} \right]%
		   }_{3}
		\frac{1}{2}
		   \left[ \begin{array}{c} 1 \\ 1 \\ 0 \end{array} \right]
		= \left[ \begin{array}{r}
			- \frac{1}{2} \medskip \\ \frac{1}{2} \medskip \\ 3
			\end{array} \right] \\
		\| v^2 \|^2 &= 9 \frac{1}{2} = \frac{19}{2} ;\\
		\\
		v^3	&= y^3 - \frac{<v^1,y^3>}{\| v^1 \|^2} v^1
			- \frac{<v^2,y^3>}{\| v^2 \|^2} v^2 \\
		&= \left[ \begin{array}{c} 0 \\ 1 \\ 1 \end{array} \right]
		- \underbrace{%
		   \left[ \begin{array}{ccc} 1 & 1 & 0 \end{array} \right]
		   \left[ \begin{array}{c} 0 \\ 1 \\ 1 \end{array} \right]%
		   }_{1}
		\frac{1}{2}
		   \left[ \begin{array}{c} 1 \\ 1 \\ 0 \end{array} \right]
		- \underbrace{%
		   \left[ \begin{array}{ccc}
			-\frac{1}{2} & \frac{1}{2} & 3 \end{array} \right]
		   \left[ \begin{array}{c} 0 \\ 1 \\ 1 \end{array} \right]%
		   }_{3\frac{1}{2}}
		\frac{1}{\frac{19}{2}}
		   \left[ \begin{array}{r}
			-\frac{1}{2} \medskip \\ \frac{1}{2} \medskip \\ 3 \end{array} \right] \\
		&= \left[ \begin{array}{c} 0 \\ 1 \\ 1 \end{array} \right]
		- \left[ \begin{array}{c}
			\frac{1}{2} \\ \frac{1}{2} \\ 0 \end{array} \right]
		- \left[ \begin{array}{r}
			-\frac{7}{38} \medskip \\ \frac{7}{38} \medskip \\ \frac{21}{19}
			\end{array} \right]
		= \left[ \begin{array}{r}
			-\frac{6}{19} \medskip \\ \frac{6}{19} \medskip \\ -\frac{2}{19}
			\end{array} \right].
	\end{align*}

\noindent\textbf{Normalize to obtain Orthonormal Basis (often useful to do this, but not always required):}\\
	\begin{align*}
		\tilde{v}_1 &= \frac{v^1}{\| v^1 \|} = \left[ \begin{array}{c} \frac{1}{\sqrt{2}} \medskip \\ \frac{1}{\sqrt{2}} \medskip \\ 0 \end{array} \right]\\
		\tilde{v}_2 &= \frac{v^2}{\| v^2 \|} = \left[ \begin{array}{r} \frac{-1}{\sqrt{38}} \medskip \\ \frac{1}{\sqrt{38}} \medskip \\ 3\sqrt{\frac{2}{19}} \end{array} \right]\\
		\tilde{v}_3 &= \frac{v^3}{\| v^3 \|} = \frac{19}{\sqrt{76}} \left[ \begin{array}{r} -\frac{6}{19} \medskip \\ \frac{6}{19} \medskip \\ -\frac{2}{19}
			\end{array} \right]\\
	\end{align*}
	

\newpage

\underline{Inner Products: Example Computation for $(C[0,1],\RR)$}

\textbf{Given data:}\\
	$$C[0,1] = \{ f:[0,1] \rightarrow \RR \mid f~~\mbox{continuous} \},\quad
		<f,g> = \int_0^1 f(\tau) g(\tau) d\tau$$ \\
	$$\{ y^1, y^2, y^3 \} = \{ 1, t, t^2 \} \\$$

\vspace*{1cm}

\textbf{Apply Gram-Schmidt to Produce an Orthogonal Basis:}\\
	\begin{align*}
		v^1 &= y^1 = 1\\
		\qquad \| v^1 \|^2 &= \int_0^1 (1)^2 d\tau = 1 ;\\
		\\
		v^2	&= y^2 - \frac{<v^1,y^2>}{\| v^1 \|^2} v^1 \\
		&= t - \underbrace{%
			\int_0^1 1 \cdot \tau d\tau%
			}_{\frac{1}{2}} \cdot \frac{1}{1} \cdot 1
		= t - \frac{1}{2} \\
		\| v^2 \|^2 &= \int_0^1 (\tau - \frac{1}{2})^2 d\tau = \frac{1}{12} ;\\
		\\
		v^3	&= y^3 - \frac{<v^1,y^3>}{\| v^1 \|^2} v^1
		- \frac{<v^2,y^3>}{\| v^2 \|^2} v^2 \\
		&= t^2 - \underbrace{%
			\int_0^1 1 \cdot \tau^2 d\tau%
			}_{\frac{1}{3}} \cdot \frac{1}{1} \cdot 1
		- \underbrace{%
			\int_0^1 (\tau - \frac{1}{2}) \tau^2 d\tau%
			}_{\frac{1}{12}} \left( \frac{1}{\frac{1}{12}} \right)
				\Bigl( t - \frac{1}{2} \Bigr) \\
		&= t^2 - \frac{1}{3} - \Bigl( t - \frac{1}{2} \Bigr) \\
		&= t^2 - t + \frac{1}{6}.
	\end{align*}

%%%%%%%%%%%%%%%%%%%%%%%%%%%%%%%%%%%%%%%%%%%%%%%%%%%%%%%%%%%%%%%%%%%%%%%%%%%%%
\newpage
\bigskip
\begin{center}\large\bf
Doing Inner products on $C[a,b]$ in MATLAB
\end{center}

\begin{alltt}
>> clear *
>> syms t % declare to be a symbolic variable

>> % INT(S,a,b) is the definite integral of S with respect to
   % its symbolic variable from a to b.  a and b are each
   % double or symbolic scalars.

>> y1=1+0*t % Otherwise MATLAB is too dumb to realize
            % that y1 is a trivial function of the symbolic
            % variable t
y1 = 1

>> y2=t;
>> y3=t^2;

% Start the G-S Procedure. Here we assume C[0,1], that is
% C[a,b], with [a,b]=[0,1]

>> v1=y1

v1=1

>> v2=y2-int(v1*y2,0,1)*v1/int(v1^2,0,1)

v2=t-1/2

>> v3=y3-int(v1*y3,0,1)*v1/int(v1^2,0,1)-
int(v2*y3,0,1)*v2/int(v2^2,0,1)

v3=t^2+1/6-t

% Next, normalize to length one

v1_tilde=v1/int(v1^2,0,1)^.5

v1_tilde=1

>> v2_tilde=v2/int(v2^2,0,1)^.5

v2_tilde=(t-1/2)*12^(1/2)

>> simplify(v2_tilde);

ans=(2*t-1)*3^(1/2)

>> v3_tilde=simplyfy(v3/int(v3^2,0,1)^.5)

v3_tilde=(6*t^2+1-6*t)*5^(1/2)
\end{alltt}

\end{document}
