
The Robotics Graduate Program was designed in 2013 and launched in 2014. The vision was to take in students from all engineering backgrounds, plus many STEM backgrounds, such as Mathematics and Physics, and prepare M.S. and Ph.D. students for fruitful careers in Industry and Academia. We knew the incoming students would have very heterogeneous preparation in Mathematics, Programming, and Hardware. To form them into a more homogeneous cohort, two first-semester courses were created, ROB 501 Mathematics for Robotics and ROB 550 Robotics Laboratory\footnote{While ROB 550 was initially charged with preparing students in the areas of embedded programming, inverse kinematics, camera calibration, path planning, encoders, motor drives, and more, later, in 2019, ROB 502 Programming for Robotics was created to relive some of the burden from ROB 550.}. After taking these two courses, the students would have adequate preparation to take courses in Sensing, Acting and Reasoning, as laid out here \url{https://robotics.umich.edu/academic-program/courses/}. The Robotics Graduate Program has worked remarkably well due to the enthusiasm of the students, staff, and faculty. \\

I accepted the charge of creating the mathematical fundamentals course, ROB 501. I interviewed the Robotics faculty in Spring 2014 to find out key topics they wanted covered. Over Summer 2014, I pared the list down and organized it into a coherent set of topics that would mostly meet the needs as identified by the faculty. To the set of topics enumerated by my colleagues, I added one additional goal, to break the students' reliance on example-based (e.g., physics-based) reasoning, and instill in them the ability to conduct \textbf{abstract reasoning}. The ability to abstract problems from a physical instantiation to a mathematical formulation provides clarity of thinking as well as being the first step in making problems amenable to algorithmic solutions. Once a student has written a problem down mathematically, they are able to search the literature for potential solutions, or understand that they have stumbled upon a new question that requires new investigations. \\

To be clear, breaking a student's reliance on physics-based reasoning really means augmenting their reasoning ability, it means giving them confidence to deal with abstract formulations of problems. The key word here is \textbf{confidence}. ROB 501 does this by doing proof after proof in every lecture. By practice and repetition, the student learns to read (relatively) complex mathematical statements, negate them, and otherwise appreciate common mathematical symbols as a precise and effective form of shorthand. To make the repetition of definitions and proofs tenable, they first practice them in the friendly environment of linear algebra. We take advantage of the fact that the vast majority of the students taking the course will be comfortable with matrix-vector arithmetic and MATLAB. While Chapter 1 introduces mathematical notation and proof techniques, these topics are not really learned until they are exercised in Chapter 2 with an abstract treatment of linear algebra, with everything proved, either in class or HW. Chapter 3 adds more abstraction, inner product spaces, but starts to provide a pay off, namely deterministic least squares problems. The students start to see how concrete optimization problems can be solved easily and precisely with the normal equations, and through HW sets, that ``practical'' problems involving functions can be solved with no essential changes to the methods. In addition, as a setup for the Kalman Filter, we derive a recursive solution to the standard (batch) least squares problem, giving us RLS.\\

Chapter 4 is meant as a mental break. While students are completing HW sets and/or exams on the previous material, we use Gram Schmidt to develop the QR Factorization, and we use orthogonal matrices and eigenvectors to derive the SVD and pause to illustrate its practical importance. While typing up these notes, I added in the LU Factorization, in part because I used it as a building block in the undergrad course, ROB 101, Computational Linear Algebra. \\

Chapter 5 sets the stage for minimum variance estimation, or non-deterministic least squares problems. We build slowly because if there is one topic with which a typical first-semester engineering graduate student struggles, it's probability. The students themselves are not to blame for this, nor are their undergraduate instructors. The topic is fundamentally very challenging. As soon we leave the confines of discrete random variables (with a finite number of possible outcomes) and enter the realm of continuous random variables, we have a choice: go down the measure theory rabbit hole or fake it. To be sure, we fake it. To clear our conscience, we are upfront about it and explain the restrictions we make to allow probabilities of events to be computed via an integral, that is, via a density and Riemann integration. This allows random vectors to be defined and the important quantities, the mean of a random vector, its covariance, and its variance. Armed with these concepts, we revisit a standard least squares problem and add a measurement noise model, leading to the Best Linear Unbiased Estimator (BLUE), and then a measurement noise model and a stochastic model for the unknown, $x$, leading to the Minimum Variance Estimator (MVE). From here, the goal is the Kalman filter. To get there, conditional probability is needed, leading to a treatment of conditional normal random vectors, which, once distilled to a set of key facts, yields the Kalman filter.\\

Chapter 6 is motivated by considering the tools required to understand the convergence of sequences and the existence of minima and maxima of functions. It's also a last pass through our proof techniques, because Chapters 4 and 5 were proof-light, being more focused on important applications of the math we had already done. We introduce the basic notions of topology, open and closed sets, in the context of normed spaces. Open and closed sets can be characterized with the notion of distance and through the convergence of sequences, showing the interplay of things that may not seem so related at a student's first glance. The contraction mapping principle is covered as are standard results on compact sets and continuous functions on compact sets. This provides general conditions that are easy to check in many engineering situations for the existence of extrema of continuous real-valued functions. These notes conclude with brief remarks on convex functions, QPs, and LPS for minimizing the one-norm and the max-norm.\\

\textbf{Jessy Grizzle}
Ann Arbor, Michigan USA