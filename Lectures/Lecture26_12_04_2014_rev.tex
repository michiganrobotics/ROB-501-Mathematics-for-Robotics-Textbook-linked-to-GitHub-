\documentclass[letterpaper]{article}
\usepackage{amssymb}
\usepackage{fullpage}
\usepackage{amsmath}
\usepackage{epsfig,float,alltt}
\usepackage{psfrag,xr}
\usepackage[T1]{fontenc}
\usepackage{url}
\usepackage{pdfpages}
\usepackage{graphicx}
%\includepdfset{pagecommand=\thispagestyle{fancy}}

%
%***********************************************************************
%               New Commands
%***********************************************************************
%
%
\newcommand{\rb}[1]{\raisebox{1.5ex}{#1}}
 \newcommand{\trace}{\mathrm{trace}}
\newcommand{\real}{\mathbb R}  % real numbers  {I\!\!R}
\newcommand{\nat}{\mathbb R}   % Natural numbers {I\!\!N}
\newcommand{\cp}{\mathbb C}    % complex numbers  {I\!\!\!\!C}
\newcommand{\ds}{\displaystyle}
\newcommand{\mf}[2]{\frac{\ds #1}{\ds #2}}
\newcommand{\book}[2]{{Luenberger, Page~#1, }{Prob.~#2}}
\newcommand{\spanof}[1]{\textrm{span} \{ #1 \}}
 \newcommand{\cov}{\mathrm{cov}}
 \newcommand{\E}{\varepsilon}
\parindent 0pt
%
%
%***********************************************************************
%
%               End of New Commands
%
%***********************************************************************
%

\begin{document}

\baselineskip=48pt  % Enforce double space

%\baselineskip=18pt  % Enforce 1.5 space
\setlength{\parskip}{.3in}
\setlength{\itemsep}{.3in}

\pagestyle{plain}

{\Large \bf
\begin{center}
Rob 501 Fall 2014\\
Lecture 26\\
Typeset by:  Vittorio Bichucher\\
Proofread by: Mia Stevens\\
Revised by Ni on Nov. 29, 2015
\end{center}
}

\Large

\begin{center}\textbf{Convex Sets and Convex Functions (Continued)}\end{center}

\noindent \textbf{Additional Facts:}~
    \begin{itemize}
        \item All norms $\|\cdot\|:X\rightarrow [0, \infty)$ are convex. (proof using triangle inequality)
        \item For all $1\leq\beta<\infty$, $\|\cdot\|^\beta$ is convex. (Convex function $\times$ strictly increasing function) Hence, on $\mathbb{R}^n$:
		    \begin{equation*}
        	    \Sigma_{i=1}^{n}|x_i|^3
            \end{equation*}
            is convex.
        \item Let $r>0$, $\|\cdot\|$ a norm, $B_r(x_0)$ is a convex set.\\ \\
            Special case: $B_1(0)$ convex set. (unit ball about the origin)\\
            Let C be an open, bounded and convex set, 0 $\in$ C. Then, $\exists\ \|\cdot\|: X \rightarrow [0,\infty)$ such that $C=\{x\in X\ |\:||x||<1\} = B_1(0)$.
        \item $K_1$ convex, $K_2$ convex $\rightarrow K_1 \cap K_2$ is convex. (Proved by line inside the set)
        \item Consider $(\mathbb{R}^n, \mathbb{R})$, $A$ is a real $m$ by $n$ matrix, $b\in \mathbb{R}^m$. Then:
        \begin{itemize}
            \item $K=\{x \in \mathbb{R}^n|\:Ax\leq b\}$ is also convex. (linear inequality)
            \item $K=\{x \in \mathbb{R}^n|\:Ax=b\}$ is convex. (linear equality)
            \item $K=\{x \in \mathbb{R}^n|\:A_{eq}x=b_{eq},\ A_{in}x\leq b_{in}\}$ is convex as well. (intersection property)
        \end{itemize}
    \end{itemize}

\textbf{Remark:} $\tilde{A}x \geq \tilde{b} \Leftrightarrow -\tilde{A}x \leq -\tilde{b}$.

\begin{center}\textbf{Quadratic Programming}\end{center}

    $x\in \real^n$, $Q\geq 0$.\\
    Minimize: $\underbrace{x^TQx}_{\textnormal{quadratic term}}+\underbrace{fx}_{\textnormal{linear term}}$ subject to $A_{in}x \leq b_{in}$ and $A_{eq}x = b_{eq}$
    
    \textbf{Note:}~ $f(x)$, $Q$, $A_{in}$ and $A_{eq}$ are all convex. Also, check if constraints form the empty set.
    
    There are special purposes solvers available! See S. Boyd's website!

    Example using robot equation:
        \begin{equation*}
            D(q)\ddot{q}+C(q,\dot{q})\dot{q}+G(q) = Bu
        \end{equation*}
    where $q \in \mathbb{R}^n$, $u\in \mathbb{R}^m$.
    
    Further, the ground reaction forces can be modeled as:
        \begin{equation*}
            F = \Lambda_0(q,\dot{q})+\Lambda_1(q)u = \left[ \begin{array}{c}
												F^h \\
                                                F^v \end{array} \right].
        \end{equation*}
        
    Suppose the desired feedback signal is $u=\gamma(q,\dot{q})$, but we need to respect bounds on the ground reaction forces
        \begin{equation*}
            F^v \geq 0.2m_{total}g.
        \end{equation*}
        
    Therefore, the normal force should be at least $20\%$ of the total weight
        \begin{equation*}
            |F^h| \leq 0.6F^v.
        \end{equation*}
        
    Therefore, the friction force has a cone shape, and its magnitude is less than $60\%$ of the total vertical force. Putting it all together:
        \begin{equation*}
            \left[ \begin{array}{c}
		        F^v \geq 0.2m_{total}g\\
                F^h \leq 0.6F^v\\
                -F^h \leq 0.6F^v\end{array} \right]
                \Leftrightarrow A_{in} (q)u\leq b_{in}(q,\dot{q}).
        \end{equation*}

    \textbf{QP:}
    \begin{equation*}
        u^* = \operatorname{argmin}\ u^Tu+d^Tdp
    \end{equation*}
    \begin{equation*}
        A_{in}(q)u\leq b_{in}(q,\dot q)
    \end{equation*}
    \begin{equation*}
        u = \gamma(q, \dot{q})+d^Td
    \end{equation*}
    where $d^Td$ is often called the relaxation parameter. Further, $p$ is an weighting factor and it should be $>>>>1\cdot10^4$. Dr. Grizzle finished by showing his handout in linear programming and quadractic programming. And remember Stephen Boyd!

\end{document} 