%**************************************************************
%References for commands and symbols:
%1. https://en.wikibooks.org/wiki/LaTeX/Mathematics
%2. http://latex.wikia.com/wiki/List_of_LaTeX_symbols
%**************************************************************

\documentclass[letterpaper]{article}
\usepackage{amssymb}
\usepackage{fullpage}
\usepackage{amsmath}
\usepackage{epsfig,float,alltt}
\usepackage{psfrag,xr}
\usepackage[T1]{fontenc}
\usepackage{url}
\usepackage{pdfpages}
%\includepdfset{pagecommand=\thispagestyle{fancy}}

%
%***********************************************************************
%               New Commands
%***********************************************************************
%
%
\newcommand{\rb}[1]{\raisebox{1.5ex}{#1}}
 \newcommand{\trace}{\mathrm{trace}}
\newcommand{\real}{\mathbb R}  % real numbers  {I\!\!R}
\newcommand{\nat}{\mathbb N}   % Natural numbers {I\!\!N}
\newcommand{\whole}{\mathbb Z}    % Integers/whole numbers  {I\!\!\!\!Z}
\newcommand{\cp}{\mathbb C}    % complex numbers  {I\!\!\!\!C}
\newcommand{\rat}{\mathbb Q}    % rational numbers  {I\!\!\!\!Q}

\newcommand{\ds}{\displaystyle}
\newcommand{\mf}[2]{\frac{\ds #1}{\ds #2}}
\newcommand{\book}[2]{{Luenberger, Page~#1, }{Prob.~#2}}
\newcommand{\spanof}[1]{\textrm{span} \{ #1 \}}
 \newcommand{\cov}{\mathrm{cov}}
 \newcommand{\E}{\mathcal{E}}
\parindent 0pt
%
%
%***********************************************************************
%
%               End of New Commands
%
%***********************************************************************
%

\begin{document}


\baselineskip=48pt  % Enforce double space

%\baselineskip=18pt  % Enforce 1.5 space

\setlength{\parskip}{.3in}
\setlength{\itemsep}{.3in}

\pagestyle{plain}

{\Large \bf
\begin{center}
Rob 501 Handout: Grizzle \\
Lecture 10\\
Orthogonal Projection and Normal Equations
\end{center}
}



\Large

\begin{center}
\textbf{Projection Theorem (Continued)}
\end{center}

\noindent \textbf{Orthogonal Projection Operator}
\newline
Let $\cal X$ be a finite dimensional (real) inner product space and $M$ a subspace of $\cal X$. For $x \in \cal X$ and $m_0 \in M$. The Projection Theorem shows the TFAE:
\renewcommand{\labelenumi}{(\alph{enumi})}
\begin{enumerate}
    \item $x-m_0  \bot M $.

    \item $\exists \tilde{m} = M^{\bot}$ such that  $ x=m_0 + \tilde{m} $.

    \item $ \|x-m_0\|=d(x,M)=\inf\limits_{m \in M} \|x-m\|$.
\end{enumerate}
\noindent \textbf{Def.}~ $P$: ${\cal X}\rightarrow M$ by $P(x)=m_0$, where $m_0$ satisfies any of (a),(b) or (c), is called the \underline{orthogonal projection} of $\cal X$ onto $M$.

\noindent \textbf{Exercise1:}
$P$: ${\cal X}\rightarrow M$ is a linear operator.

\noindent \textbf{Exercise2:}
$P$: Let $\{v^1,\cdots ,v^k\}$ be an orthonormal basis for $M$.Then
$$P(x)= \sum\limits_{i=1}^k  \langle x,v^i \rangle  v^i.$$

\newpage

\noindent \textbf{Normal Equations}
\newline
Let $\cal X$ be a finite dimensional (real) inner product space and $M=\mathrm{span}\{y^1,\cdots ,y^k\}$, with $\{y^1,\cdots ,y^k\}$ linearly independent. Given $x \in \cal X$, seek $\hat{x} \in M$ such that
$$\|x-\hat{x} \|= d(x,M)=\inf_{m \in M} \|x-m\|= \min_{m \in M} \|x-m\|$$
where we can write ``$\min$'' because the Projection Theorem assures the existence of a minimizing vector $\hat{x} \in M$.

\noindent \textbf{Notation:} $\hat{x} = \operatorname*{argmin}d(x,M)$

\noindent \textbf{Remark:} One solution is Gram Schmidt and the orthogonal projection operator.
We provide an alternative way to compute the answer.

By the Projection Theorem, $\hat{x} $ exists and is characterized by $x-\hat{x} \bot M$.
Write
$$\hat{x}= \alpha_1 y^1+ \alpha_2 y^2+ \cdots +\alpha_k y^k$$
and impose $x-\hat{x} \bot M \Leftrightarrow x-\hat{x} \bot y^i,~ 1 \le i \le k$.

Then, $ \langle x-\hat{x},y^i \rangle =0,~~\forall 1 \leq i \leq k$ yields
\begin{align*}
&\langle \hat{x},y^i \rangle = \langle x,y^i \rangle     ~~  i=1,2,\cdots, k \\
\Leftrightarrow&\langle \alpha_1 y^1+ \alpha_2 y^2+\cdots +\alpha_k y^k, y^i \rangle  =  \langle x,y^i \rangle      ~~   i=1,2,\cdots ,k.
\end{align*}

We now write this out in matrix form.\\
$\underline{i=1}$
$$\alpha_1  \langle y^1,y^1 \rangle +\alpha_2  \langle y^2,y^1 \rangle + \cdots + \alpha_k  \langle y^k,y^1 \rangle =  \langle x,y^1 \rangle $$
$\underline{i=2}$
$$\alpha_1  \langle y^1,y^2 \rangle +\alpha_2  \langle y^2,y^2 \rangle + \cdots + \alpha_k  \langle y^k,y^2 \rangle =  \langle x,y^2 \rangle $$ \\
$$\vdots$$\\
$\underline{i=k}$
$$\alpha_1  \langle y^1,y^k \rangle +\alpha_2  \langle y^2,y^k \rangle + \cdots + \alpha_k  \langle y^k,y^k \rangle =  \langle x,y^k \rangle $$

\noindent These are called \underline{the Normal Equations}.

\noindent \textbf{Def.}
$G=G(y^1,\cdots , y^k)=\left[ \begin{array}{cccc}  \langle y^1,y^1 \rangle  &  \langle y^1,y^2 \rangle  & \cdots &  \langle y^1,y^k \rangle  \\  \langle y^2,y^1 \rangle  &  \langle y^2,y^2 \rangle  & \cdots &  \langle y^2,y^k \rangle \\ \vdots & \vdots && \vdots \\  \langle y^k,y^1 \rangle  &  \langle y^k,y^2 \rangle  & \cdots &  \langle y^k,y^k \rangle  \end{array}  \right]$

$G_{ij}= \langle y^i,y^j \rangle $ is called \underline{the Gram matrix}.\\

\noindent \textbf{Remark:} Because we are assuming $\cal F =\mathbb{R} $,  $ \langle y^i,y^j \rangle = \langle y^j,y^i \rangle $, and we therefore have $G=G^T$.

Let $\alpha=\begin{bmatrix} \alpha_1 \\ \alpha_2 \\ \vdots \\ \alpha_k  \end{bmatrix}$,  we have
\begin{center}
$G^T \alpha=\beta $ (normal equation in the matrix form)\\
\end{center}
where
$$\beta_i= \langle x,y^i \rangle ,\quad \beta=\begin{bmatrix} \beta_1 \\ \beta_2 \\ \vdots \\ \beta_k  \end{bmatrix}.$$

\newpage

\noindent \textbf{Def.}~ $g(y^1,y^2,\cdots,y^k)=$det $G(y^1,\cdots,y^k)$ is the determinant of the Gram Matrix.

\noindent \textbf{Prop.}~ $g(y^1,y^2,\cdots,y^k) \neq 0 \Leftrightarrow \{ y^1,\cdots, y^k \}$ is linearly independent.

\noindent The proof is given at the end of the handout.

\noindent \textbf{Summary:}~ Here is the solution of our best approximation problem by the normal equations. Assume the set $\{ y^1,\cdots, y^k \}$ is linearly independent and $M:=\spanof{y^1,\cdots, y^k }$. Then
$\hat{x} = \text{arg~min}~ d(x,M)$ if, and only if,
\begin{align*}
\hat{x} &= \alpha_1 y^1+ \alpha_2 y^2+ \cdots +\alpha_k y^k \\
\\
G^T \alpha &=\beta \\
\\
G_{ij}&= \langle y^i,y^j \rangle \\
\\
\beta_i &= \langle x,y^i \rangle.
\end{align*}

\newpage

\noindent \textbf{Application:}~Over determined system of linear equations in $\mathbb{R}^n$
$$A \alpha=b,$$
where $A=n \times m$ real matrix,  $n \geq m$, rank$(A)=m$ (columns of $A$ are linearly independent). From the dimension of $A$, we have that $\alpha \in \mathbb{R}^m , b \in \mathbb{R}^n$.

\underline{Original Problem Formulation:}\\
Seek $\hat{\alpha} $ such that $$\| A \hat{\alpha}-b \|=\min_{\alpha \in \mathbb{R}^m} \| A \alpha -b \|,$$
where $$\|x\|^2 =\sum\limits_{i=1}^n (x_i)^2. $$

\underline{Solution:}
$${\cal X}=\mathbb{R}^n, \quad {\cal F}=\mathbb{R}, \quad  \langle x,y \rangle = x^T y= y^T x= \sum\limits_{i=1}^n x_i y_i$$
Therefore,  $$\|x\|^2= \langle x,x \rangle =\sum\limits_{i=1}^n |x_i|^2.$$
Write $$A=[A_1 | A_2 | \cdots | A_m] ~\text{and}~\alpha = [\alpha_1, \alpha_2,  \cdots, \alpha_m]^\top$$
and we note that
$$ A \alpha = \alpha_1 A_1 + \alpha_2 A_2 + \cdots \alpha_m A_m.$$

\underline{New Problem Formulation:}\\
Seek $$\hat{x} = A \hat{\alpha} \in {\rm span}\{ A_1,A_2, \cdots, A_m \} =: M$$
such that $$ \|\hat{x}-b\|=d(b,M) \Leftrightarrow \hat{x}-b \bot M.$$

\newpage

From the Projection Theorem and the Normal Equations,
$$\hat{x}=\hat{\alpha}_1 A_1 +\hat{\alpha}_2 A_2 + \cdots \hat{\alpha}_m A_m $$
and $G^\top \hat{\alpha}=\beta$, with
$$G_{ij}= \langle A_i,A_j \rangle =A_i ^\top A_j$$ $$\beta_i =  \langle b,A_i \rangle  = b^\top A_i = A_i^\top b.$$


%\underline{Objective}$$\min_{\alpha \in \mathbb{R}^m} \|A \alpha -b \|=\| \alpha_1 A_1+\alpha_2 A_2+ \cdots +\alpha_m A_m -b \|$$

\underline{Aside}
$$A^\top=\begin{bmatrix} A_1^\top \\ A_2^\top \\ \vdots \\ A_m^\top  \end{bmatrix} \qquad A=[A_1| \cdots | A_m ]$$ $$(A^\top A)_{ij}=A^\top_i A_j$$
$$G=G^\top=A^\top A$$ $$(A^\top b)_i=A_i^\top b$$

Normal Equations are $$A^\top A \hat{\alpha} =A^\top b.$$
From the Proposition, $G^\top =A^\top A $ is invertible $ \Leftrightarrow $  columns of $A$ are linearly independent. Hence, $$\hat{\alpha}=(A^\top A)^{-1} A^\top b.$$

%\newpage
%
%\noindent \textbf{Notes:}
%\begin{enumerate}    \item Appropriate inner product space determined by the columns of $A$ and the error term $\|A \alpha-b\|$. The inner product space is not determined by $\alpha$.
%
%    \item What is the approximation error?
%    \begin{align*}
%    \hat{x}&=A \hat{\alpha} \\
%    &= A(A^\top A)^{-1} A^\top b\\
%    \\
%    e&=\hat{x}-b \\
%    &=A (A^\top A)^{-1} A^\top b-b\\
%    &=\left(A (A^\top A)^{-1} A^\top -I\right)b
%    \end{align*}

%
%\noindent    $\hat{x}$ is the orthogonal projection of $b$ onto the span of the columns of $A$
%\end{enumerate}

\newpage

\noindent \textbf{Prop.}~ $g(y^1,y^2,\cdots,y^k) \neq 0 \Leftrightarrow \{ y^1,\cdots, y^k \}$ is linearly independent.\\

\noindent \textbf{Proof:} $g(y^1,y^2,\cdots,y^k) = 0 \leftrightarrow \exists \alpha \neq 0~~\text{such that}~ G^\top \alpha =0$.

From our construction of the normal equations,  $G^\top \alpha = 0$ if, and only if
$$  \langle \alpha_1 y^1+ \alpha_2 y^2+\cdots +\alpha_k y^k, y^i \rangle  = 0     ~~   i=1,2,\cdots ,k.$$
This is equivalent to
$$  \left( \alpha_1 y^1+ \alpha_2 y^2+ \cdots +\alpha_k y^k \right) \bot  y^i = 0     ~~   i=1,2,\cdots ,k$$
which is equivalent to
$$  \left( \alpha_1 y^1+ \alpha_2 y^2+ \cdots + \alpha_k y^k \right) \bot  \textrm{span} \{ y^1, \cdots, y^k \}=:M $$
and thus
$$\left( \alpha_1 y^1+ \alpha_2 y^2+ \cdots + \alpha_k y^k \right) \in M^\perp.$$

Because $\alpha_1 y^1+ \alpha_2 y^2+\cdots +\alpha_k y^k \in M$, we have that
$$\left( \alpha_1 y^1+ \alpha_2 y^2+\cdots +\alpha_k y^k  \right) \in M \cap M^\perp $$
and therefore
$$  \alpha_1 y^1+ \alpha_2 y^2+\cdots +\alpha_k y^k  = 0. $$ By the linear independence of
$\{ y^1,\cdots, y^k \}$, we deduce that $$\alpha_1 = \alpha_2 = \cdots \alpha_k = 0.\ \square$$


\end{document}



