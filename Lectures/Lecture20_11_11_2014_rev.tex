%**************************************************************
%References for commands and symbols:
%1. https://en.wikibooks.org/wiki/LaTeX/Mathematics
%2. http://latex.wikia.com/wiki/List_of_LaTeX_symbols
%**************************************************************

\documentclass[letterpaper]{article}
\usepackage{amssymb}
\usepackage{fullpage}
\usepackage{amsmath}
\usepackage{epsfig,float,alltt}
\usepackage{psfrag,xr}
\usepackage[T1]{fontenc}
\usepackage{url}
\usepackage{pdfpages}
\usepackage{mathrsfs}
%\includepdfset{pagecommand=\thispagestyle{fancy}}

%
%***********************************************************************
%               New Commands
%***********************************************************************
%
%
\newcommand{\rb}[1]{\raisebox{1.5ex}{#1}}
 \newcommand{\trace}{\mathrm{trace}}
\newcommand{\real}{\mathbb R}  % real numbers  {I\!\!R}
\newcommand{\nat}{\mathbb N}   % Natural numbers {I\!\!N}
\newcommand{\whole}{\mathbb Z}    % Integers/whole numbers  {I\!\!\!\!Z}
\newcommand{\cp}{\mathbb C}    % complex numbers  {I\!\!\!\!C}
\newcommand{\rat}{\mathbb Q}    % rational numbers  {I\!\!\!\!Q}

\newcommand{\ds}{\displaystyle}
\newcommand{\mf}[2]{\frac{\ds #1}{\ds #2}}
\newcommand{\book}[2]{{Luenberger, Page~#1, }{Prob.~#2}}
\newcommand{\spanof}[1]{\textrm{span} \{ #1 \}}
 \newcommand{\cov}{\mathrm{cov}}
 \newcommand{\E}{\mathcal{E}}
\parindent 0pt
%
%
%***********************************************************************
%
%               End of New Commands
%
%***********************************************************************
%

\begin{document}

\baselineskip=48pt  % Enforce double space

%\baselineskip=18pt  % Enforce 1.5 space

\setlength{\parskip}{.3in}
\setlength{\itemsep}{.3in}

\pagestyle{plain}

{\Large \bf
\begin{center}
Rob 501 Fall 2014\\
Lecture 20\\
Typeset by:  Yevgeniy Yesilevskiy\\
Revised by Ni on 21 Nov. 2015
\end{center}
}

\Large

\begin{center}
    \textbf{Multivariate Random Variables or Vectors}
\end{center}

\noindent Let $(\Omega,\ \mathscr{F},\ P)$ be a \underline{probability space}.
    \begin{equation*}
        X = \begin{bmatrix}
            X_1\\
            X_2
        \end{bmatrix}
    \end{equation*}
    where $X_1\in\mathbb{R}^n$ and $X_2\in\mathbb{R}^m$, and let $p=n+m$.\\
    Then, \underline{the distribution function}
    \begin{align*}
        F_{X_1X_2}(x_1,\ x_2)&=P(X_1\leq x_1,\ X_2\leq x_2)\\
        &=P(\{\omega\in\Omega|X_1(\omega)\leq x_1,\ X_2(\omega)\leq x_2\})
    \end{align*}

\noindent \textbf{Conditioning:}
    \begin{align*}
        F_{X_1|X_2}(x_1|x_2)&=P(X_1\leq x_1|X_2=x_2)\\
        &=\lim_{\epsilon\rightarrow 0}\frac{P(A\cap B_\epsilon)}{P(B_\epsilon)}
    \end{align*}
    where $A=\{\omega|X_1(\omega)\leq x_1\}$, $B_\epsilon=\{\omega|x_2-\epsilon\leq X_2(\omega)\leq x_2+\epsilon\}$

\noindent \textbf{Conditional Density:}
    \begin{equation*}
        f_{X_1|X_2}=\frac{f_{X_1X_2}(x_1,\ x_2)}{f_{X_2}(x_2)}
    \end{equation*}
    Sometimes, it is convenient to write $f(x_1|x_2)$.

\noindent \textbf{Conditional Mean (Expectation):}
    \begin{align*}
        \mu(x_2)&=E\{X_1|X_2=x_2\}=\int_{\mathbb{R}^n}x_1f(x_1|x_2)dx_1\\
        &=\int_{\mathbb{R}^n}x_1f_{X_1|X_2}(x_1|x_2)dx_1
    \end{align*}

\noindent \textbf{Theorem:} Let $\hat{x}=\operatorname*{argmin}_{z=g(x_2)}E\{\|X_1-z\|^2|X_2=x_2\}$, where $g$ varies over all functions $g:\ \mathbb{R}^m\rightarrow\mathbb{R}^n$.\\
    Then, $\hat{x}=\mu(x_2)=E\{X_1|X_2=x_2\}$.

\noindent \textbf{Remark:} $g:\ \mathbb{R}^m\rightarrow\mathbb{R}^n$ includes linear, quadratic, cubic ... terms.

\end{document}
