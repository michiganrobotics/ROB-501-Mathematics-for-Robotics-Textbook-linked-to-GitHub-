%**************************************************************
%References for commands and symbols:
%1. https://en.wikibooks.org/wiki/LaTeX/Mathematics
%2. http://latex.wikia.com/wiki/List_of_LaTeX_symbols
%**************************************************************

\documentclass[letterpaper]{article}
\usepackage{amssymb}
\usepackage{fullpage}
\usepackage{amsmath}
\usepackage{epsfig,float,alltt}
\usepackage{psfrag,xr}
\usepackage[T1]{fontenc}
\usepackage{url}
\usepackage{pdfpages}
%\includepdfset{pagecommand=\thispagestyle{fancy}}

%
%***********************************************************************
%               New Commands
%***********************************************************************
%
%
\newcommand{\rb}[1]{\raisebox{1.5ex}{#1}}
 \newcommand{\trace}{\mathrm{trace}}
\newcommand{\real}{\mathbb R}  % real numbers  {I\!\!R}
\newcommand{\nat}{\mathbb N}   % Natural numbers {I\!\!N}
\newcommand{\whole}{\mathbb Z}    % Integers/whole numbers  {I\!\!\!\!Z}
\newcommand{\cp}{\mathbb C}    % complex numbers  {I\!\!\!\!C}
\newcommand{\rat}{\mathbb Q}    % rational numbers  {I\!\!\!\!Q}

\newcommand{\ds}{\displaystyle}
\newcommand{\mf}[2]{\frac{\ds #1}{\ds #2}}
\newcommand{\book}[2]{{Luenberger, Page~#1, }{Prob.~#2}}
\newcommand{\spanof}[1]{\textrm{span} \{ #1 \}}
 \newcommand{\cov}{\mathrm{cov}}
 \newcommand{\E}{\mathcal{E}}
\parindent 0pt
%
%
%***********************************************************************
%
%               End of New Commands
%
%***********************************************************************
%

\begin{document}


\baselineskip=48pt  % Enforce double space

%\baselineskip=18pt  % Enforce 1.5 space

\setlength{\parskip}{.3in}
\setlength{\itemsep}{.3in}

\pagestyle{plain}

{\Large \bf
\begin{center}
Rob 501 Fall 2014\\
Lecture 21\\
Typeset by:  Jeff Koller\\
Proofread by: Yevgeniy Yesilevskiy\\
Revised by Grizzle on 10 Nov. 2015
\end{center}
}

\Large

\begin{center}\textbf{Luenberger Observers}\end{center}

\textbf{Luenberger Observers:}~ It is deterministic estimator. We consider the easiest case

$$x_{k+1} = Ax_k$$
$$y_k = Cx_k$$

where $x \in \real^n$, $y \in \real^p$, $A \in \real^{n \times n}$, and $C \in \real^{p \times n}$.

\textbf{Question 1:}~ When can we reconstruct the initial condition $\left( x_o \right)$ from the measurements $y_0, y_1, y_2, \ldots$
\begin{align*}
y_o &= Cx_o\\
y_1 &= Cx_1 = CAx_o\\
y_2 &= Cx_2 = CAx_1 = CA^2x_o\\
&\vdots\\
y_k &= CA^kx_o
\end{align*}

Represent the above matrix form:

$$ \left[ \begin{array}{c} y_o \\ y_1 \\ \vdots \\ y_k \end{array} \right] = \left[ \begin{array}{c} C \\ CA \\ \vdots \\ CA^k \end{array} \right]x_o$$

We note that if $\text{rank} \left[ \begin{array}{c} C \\ CA \\ \vdots \\ CA^k \end{array} \right] = n$, then we can determine $x_0$ uniquely on the basis of the measurements.

\textbf{Caley Hamilton Theorem:}

$$\text{rank} \left[ \begin{array}{c} C \\ CA \\ \vdots \\ CA^{n-1} \end{array} \right] = \text{rank} \left[ \begin{array}{c} C \\ CA \\ \vdots \\ CA^{k} \end{array} \right] \text{ for all } k \ge n-1$$

\textbf{Theorem:}~ $\text{rank} \left[ \begin{array}{c} C \\ CA \\ \vdots \\ CA^{n-1} \end{array} \right] = n$
means that we can determine $x_o$ uniquely from the measurements. (This called the Kalman observability rank condition.) \\ \\

\textbf{Question 2:}~ Can we process the measurements dynamically (i.e. recursively) and ``estimate'' $x_k$?

\textbf{Full-State Luenberger Observer:}
$$\hat{x}_{k+1} = A\hat{x}_k + L(y_k - C\hat{x}_k)$$

We define the error to be $e_k = x_k - \hat{x}_k$. We want conditions such that $e_k \to 0$ as  $k \rightarrow \infty$. Want $e_k \rightarrow 0$ because then $\hat{x}_k \rightarrow x_k$!!!

\begin{align*}
e_{k+1} & = x_{k+1} - \hat{x}_{k+1}\\
& = Ax_k - \left[A\hat{x}_k+L(y_k - C\hat{x}_k)\right]\\
& = A(x_k - \hat{x}_k) - LC(x_k - \hat{x}_k)\\
& = Ae_k-LCe_k\\
\end{align*}
$$\boxed{e_{k+1} = (A-LC)e_k}$$



\textbf{Theorem:}~ Let $e_0 \in \real^n$ and define $e_{k+1} = (A-LC)e_k$. The the sequence $e_k \rightarrow 0$ as $k \rightarrow \infty $ for all $e_0 \in \real^n$ if, and only if, $|\lambda_i(A-LC) | < 1$ for $i = 1,\ldots,n$.

\textbf{Theorem:}~ A sufficient condition for the existence of $L:\real^m \rightarrow \real^n$ that places eigenvalues of $(A-LC)$ in the unit circle is:

$$\text{rank} \left[ \begin{array}{c} C \\ CA \\ \vdots \\ CA^{n-1} \end{array} \right] = n = dim(x)$$.

\textbf{Remarks:}~ L = constant similar to $K_{ss}$ = steady-state Kalman Gain
\begin{enumerate}
\item Reason to choose one gain over the other: Optimality of the estimate when you know the noise statistics.
\item Kalman Filter works for time varying models $A_k,C_k,G_k,$ etc.
\end{enumerate}


\end{document} 