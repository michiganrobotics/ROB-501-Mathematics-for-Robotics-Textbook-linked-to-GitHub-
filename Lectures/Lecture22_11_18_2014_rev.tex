%**************************************************************
%References for commands and symbols:
%1. https://en.wikibooks.org/wiki/LaTeX/Mathematics
%2. http://latex.wikia.com/wiki/List_of_LaTeX_symbols
%**************************************************************

\documentclass[letterpaper]{article}
\usepackage{amssymb}
\usepackage{fullpage}
\usepackage{amsmath}
\usepackage{epsfig,float,alltt}
\usepackage{psfrag,xr}
\usepackage[T1]{fontenc}
\usepackage{url}
\usepackage{pdfpages}
%\includepdfset{pagecommand=\thispagestyle{fancy}}

%
%***********************************************************************
%               New Commands
%***********************************************************************
%
%
\newcommand{\rb}[1]{\raisebox{1.5ex}{#1}}
 \newcommand{\trace}{\mathrm{trace}}
\newcommand{\real}{\mathbb R}  % real numbers  {I\!\!R}
\newcommand{\nat}{\mathbb N}   % Natural numbers {I\!\!N}
\newcommand{\whole}{\mathbb Z}    % Integers/whole numbers  {I\!\!\!\!Z}
\newcommand{\cp}{\mathbb C}    % complex numbers  {I\!\!\!\!C}
\newcommand{\rat}{\mathbb Q}    % rational numbers  {I\!\!\!\!Q}

\newcommand{\ds}{\displaystyle}
\newcommand{\mf}[2]{\frac{\ds #1}{\ds #2}}
\newcommand{\book}[2]{{Luenberger, Page~#1, }{Prob.~#2}}
\newcommand{\spanof}[1]{\textrm{span} \{ #1 \}}
 \newcommand{\cov}{\mathrm{cov}}
 \newcommand{\E}{\mathcal{E}}
\parindent 0pt
%
%
%***********************************************************************
%
%               End of New Commands
%
%***********************************************************************
%

\begin{document}


\baselineskip=48pt  % Enforce double space

%\baselineskip=18pt  % Enforce 1.5 space

\setlength{\parskip}{.3in}
\setlength{\itemsep}{.3in}

\pagestyle{plain}

{\Large \bf
\begin{center}
Rob 501 Fall 2014\\
Lecture 22\\
Typeset by Ni on 18 Nov. 2015
\end{center}
}

\Large

\begin{center}
    \textbf{Real Analysis}
\end{center}

\noindent Let $(\mathcal{X},\ \real,\ \|\cdot\|)$ be a real normed space.\\
Recall $\|\cdot\|:\ \mathcal{X}\rightarrow[0,\ +\infty)$ such that
    \begin{enumerate}
        \item $\|x\|\geq 0$ and $\|x\|=0\ \Leftrightarrow\ x=0$
        \item $\|\alpha\cdot x\|=|\alpha|\cdot\|x\|$ for all $\alpha\in\real$, $x\in\mathcal{X}$
        \item $\|x+y\|\leq\|x\|+\|y\|$ for all $x,\ y\in\mathcal{X}$.
    \end{enumerate}
Recall:\\
\textbf{Def.}
    \begin{enumerate}
        \item For $x,\ y\in\mathcal{X}$, $d(x,\ y):=\|x-y\|$.
        \item For $x\in X$, $S\subset\mathcal{X}$ a subset
            \begin{equation*}
                d(x,\ S):=\inf_{y\in S}\|x-y\|.
            \end{equation*}
    \end{enumerate}

\noindent \textbf{Def.}~ Let $x_0\in X$ and $a\in\real$, $a>0$. The \underline{open ball of radius $a$ center at $x_0$} is
    \begin{equation*}
        B_a(x_0)=\{x\in\mathcal{X}|\|x-x_0\|<a\}.
    \end{equation*}

\noindent \textbf{Examples:}~
    \begin{enumerate}
        \item $\left(\real^2,\ \|\cdot\|_2\right)$: Euclidean norm
        \item $\left(\real^2,\ \|\cdot\|_1\right)$: One norm
            \begin{equation*}
                \|(x_1,\ x_2)\|_1=|x_1|+|x_2|
            \end{equation*}
        \item $\left(\real^2,\ \|\cdot\|_\infty\right)$ : Max norm
            \begin{equation*}
                \|\cdot\|_\infty=\max_{1\leq i\leq n}|x_i|
            \end{equation*}
    \end{enumerate}

\noindent \textbf{Lemma:}~ Let $\left(\mathcal{X},\ \|\cdot\|\right)$ be a normed space, $x\in\mathcal{X}$, and $S\subset\mathcal{X}$. Then,
    \begin{align*}
        d(x,\ S)=0 &\Leftrightarrow \forall\epsilon>0,\ \exists y\in S,\ \|x-y\|<\epsilon\\
        &\Leftrightarrow \forall\epsilon>0,\ B_\epsilon(x)\cap S\neq \varnothing.
    \end{align*}

\noindent \textbf{Corollary:}
    \begin{align*}
        d(x,\ S)>0 &\Leftrightarrow \exists\epsilon>0,\ \forall y\in S,\ \|x-y\|\geq\epsilon\\
        &\Leftrightarrow \exists\epsilon>0 \textnormal{ such that } B_\epsilon(x)\cap S = \varnothing
    \end{align*}
    In the following, we assume $\left(\mathcal{X},\ \|\cdot\|\right)$ is given.

\noindent \textbf{Def.}~
    \begin{enumerate}
        \item Let $P\subset\mathcal{X}$, a subset of $\mathcal{X}$. A point $p\in P$ is \underline{an interior point of $P$} if $\exists\epsilon>0$ such that $B_\epsilon(p)\subset P$.
        \item \begin{align*}
                \mathring{P}&=\{p\in P\ |\ p\textnormal{ is an interior point}\}\\
                &=\{p\in P\ |\ \exists\epsilon>0\textnormal{ such that } B_\epsilon(p)\subset P\}
            \end{align*}
            Remark for later use: $p\in \mathring{P}\Leftrightarrow\exists\epsilon>0,\ B_\epsilon(p)\subset P\Leftrightarrow\exists\epsilon>0$ such that $B_\epsilon(p)\cap(\sim P)=\varnothing\Leftrightarrow d(p,\ \sim P)>0$
            \begin{equation*}
                \sim P = P^C=\textnormal{complement}=\{x\in\mathcal{X}|x\notin P\}
            \end{equation*}
        \item $P$ is open if $P=\mathring{P}$. (Every point in $P$ is an interior point.)
    \end{enumerate}

\noindent \textbf{Proposition:}~ $x\in \mathring{P}\Leftrightarrow d(x,\ \sim P)>0$

\noindent \textbf{Example:}
    \begin{itemize}
        \item $P=(0,\ 1)\subset(\real,\ \|\cdot\|)$ is open
            \begin{align*}
                x&\in P,\ 0<x\leq\frac{1}{2},\ \epsilon=\frac{x}{2},\ B_\epsilon(x)\subset P\textnormal{, and}\\
                x&\in P,\ \frac{1}{2}\leq x<1,\ \epsilon=1-\frac{x}{2},\ B_\epsilon(x)\subset P.
            \end{align*}
        \item $P=[0,\ 1)\subset(\real,\ |\cdot|)$ is not open because $0\in P$, $\forall\epsilon>0$, $B_\epsilon(0)\cap(\sim P)\neq\varnothing$ or $0\in P$, $d(0,\ \sim P)=0$.
    \end{itemize}

\noindent \textbf{Def.}
    \begin{enumerate}
        \item A point $x\in \mathcal{X}$ is a \underline{closure point} of $P$ if $\forall\epsilon>0$, $\exists p\in P$ such that dis$\|x-p\|<\epsilon$, $[d(x,\ P)=0]$.
        \item \begin{align*}
                \textnormal{\underline{Closure of P}}=\overline{~P~}:&=\{x\in\mathcal{X}\ |\ \mathcal{X}\textnormal{ is a closure point}\}\\
                &=\{x\in\mathcal{X}\ |\ d(x,\ P)=0\}
            \end{align*}
        \item $P$ is \underline{closed} if $P=\overline{~P~}$.
    \end{enumerate}

\noindent \textbf{Example:}
    \begin{enumerate}
        \item $P=\{x\in[0,\ 1]\ |\ x\textnormal{ rational}\}\Rightarrow\overline{~P~}=[0,\ 1]$
        \item $P=(0,\ 1)\Rightarrow\overline{~P~}=[0,\ 1]$
    \end{enumerate}

\noindent \textbf{Proposition:}
    \begin{align*}
        &x\in\mathcal{X},\ x\in\overline{~P~}\Leftrightarrow d(x,P)=0.\\
        &x\in\mathcal{X},\ x\in\mathring{P}\Leftrightarrow d(x,\sim P)>0.
    \end{align*}

\noindent \textbf{Proposition:}
    \begin{align*}
        &P\textnormal{ is closed}\Leftrightarrow P=\overline{~P~}.\\
        &P\textnormal{ is open}\Leftrightarrow P=\mathring{P}.
    \end{align*}

\noindent \textbf{Proposition:}
    \begin{align*}
        &P\textnormal{ is closed}\Leftrightarrow\ \sim P\textnormal{ is open}.\\
        &P\textnormal{ is open}\Leftrightarrow\ \sim P\textnormal{ is closed}.
    \end{align*}

\noindent \underline{Proof:}
\begin{equation*}
    \underbrace{\sim P=\sim(\mathring{P})}_{P\textnormal{ is open}}=\{x\in\mathcal{X}\ |\ d(x,\ \sim P)=0\}=\underbrace{\overline{\sim P}=\sim P}_{\sim P\textnormal{ is closed}}\ \square
\end{equation*}

\end{document} 